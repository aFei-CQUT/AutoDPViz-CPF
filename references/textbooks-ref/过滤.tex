% Options for packages loaded elsewhere
\PassOptionsToPackage{unicode}{hyperref}
\PassOptionsToPackage{hyphens}{url}
%
\documentclass[
]{article}
\usepackage{amsmath,amssymb}
\usepackage{lmodern}
\usepackage{iftex}
\ifPDFTeX
  \usepackage[T1]{fontenc}
  \usepackage[utf8]{inputenc}
  \usepackage{textcomp} % provide euro and other symbols
\else % if luatex or xetex
  \usepackage{unicode-math}
  \defaultfontfeatures{Scale=MatchLowercase}
  \defaultfontfeatures[\rmfamily]{Ligatures=TeX,Scale=1}
\fi
% Use upquote if available, for straight quotes in verbatim environments
\IfFileExists{upquote.sty}{\usepackage{upquote}}{}
\IfFileExists{microtype.sty}{% use microtype if available
  \usepackage[]{microtype}
  \UseMicrotypeSet[protrusion]{basicmath} % disable protrusion for tt fonts
}{}
\makeatletter
\@ifundefined{KOMAClassName}{% if non-KOMA class
  \IfFileExists{parskip.sty}{%
    \usepackage{parskip}
  }{% else
    \setlength{\parindent}{0pt}
    \setlength{\parskip}{6pt plus 2pt minus 1pt}}
}{% if KOMA class
  \KOMAoptions{parskip=half}}
\makeatother
\usepackage{xcolor}
\usepackage{longtable,booktabs,array}
\usepackage{multirow}
\usepackage{calc} % for calculating minipage widths
% Correct order of tables after \paragraph or \subparagraph
\usepackage{etoolbox}
\makeatletter
\patchcmd\longtable{\par}{\if@noskipsec\mbox{}\fi\par}{}{}
\makeatother
% Allow footnotes in longtable head/foot
\IfFileExists{footnotehyper.sty}{\usepackage{footnotehyper}}{\usepackage{footnote}}
\makesavenoteenv{longtable}
\setlength{\emergencystretch}{3em} % prevent overfull lines
\providecommand{\tightlist}{%
  \setlength{\itemsep}{0pt}\setlength{\parskip}{0pt}}
\setcounter{secnumdepth}{-\maxdimen} % remove section numbering
\ifLuaTeX
  \usepackage{selnolig}  % disable illegal ligatures
\fi
\IfFileExists{bookmark.sty}{\usepackage{bookmark}}{\usepackage{hyperref}}
\IfFileExists{xurl.sty}{\usepackage{xurl}}{} % add URL line breaks if available
\urlstyle{same} % disable monospaced font for URLs
\hypersetup{
  hidelinks,
  pdfcreator={LaTeX via pandoc}}

% 添加中文支持
\usepackage[UTF8, fontset=windows]{ctex}  % 启用中文支持,并使用 Windows 系统的默认字体

% 设置西文字体(如果需要的话,可以修改为其他字体)
\setmainfont{Times New Roman}  % 设置英文正文字体为 Times New Roman

% 设置数学字体
\setmathfont{Latin Modern Math}  % 设置数学字体

\author{}
\date{}

\begin{document}

3.3 过滤

过滤是分离悬浮液最普遍和最有效的单元操作之一. 藉过滤操作可获得清净的液体或固相产品. 与沉降分离相比,过滤操作可使悬浮液的分离更迅速更彻底. 在某些场合下,过滤是沉降的后继操作. 过滤与蒸发,干燥等非机械操作相比,能量消耗比较低. 

3.3.1 过滤操作原理

过滤是以某种多孔物质为介质,在外力作用下,使悬浮液中的液体通过介质的孔道,而固体颗粒被截留在介质上,从而实现固,液分离的操作. 过滤操作采用的多孔物质称为过滤介质,所处理的悬浮液称为滤浆或料浆,通过多孔通道的液体称为滤液,被截留的固体物质称为滤饼或滤渣.

实现过滤操作的外力可以是重力,压力差或惯性离心力. 在化工中应用最多的还是以压力差为推动力的过滤. 

1.过滤方式

工业上的过滤操作分为两大类,即饼层过滤和深床过滤. 饼层过滤时,悬浮液置于过滤介质的一侧,固体物沉积于介质表面而形成滤饼层. 过滤介质中微细孔道的直径可能大于悬浮液中部分颗粒的直径,因而,过滤之初会有一些细小颗粒穿过介质而使滤液浑浊,但是颗粒会在孔道中迅速地发生"架桥"现象,使小于孔道直径的细小颗粒也能被截拦,故当滤饼开始形成,滤液即变清,此后过滤才能有效地进行. 可见,在饼层过滤中,真正发挥截拦颗粒作用的主要是滤饼层而不是过滤介质. 通常,过滤开始阶段得到的浑浊液,待滤饼形成后应返回滤浆槽重新处理. 饼层过滤适用于处理固体含量较高(固相体积分数约在\(1\%\) 以上)的悬浮液. 

在深床过滤中,固体颗粒并不形成滤饼,而是沉积于较厚的粒状过滤介质床层内部. 悬浮液中的颗粒尺寸小于床层孔道直径,当颗粒随流体在床层内的曲折孔道中流过时,便附在过滤介质上. 这种过滤适用于生产能力大而悬浮液中颗粒小,含量甚微(固相体积分数在\(0.1\%\)以下)的场合. 自来水厂饮水的净化及从合成纤维纺丝液中除去极细固体物质等均采用这种过滤方法. 

另外,随着膜分离技术应用领域的扩大,作为精密分离技术的膜过滤(包括微孔过滤,超滤和纳滤等)近年来发展非常迅速. 

化工中所处理的悬浮液固相浓度往往较高,故本节只讨论饼层过滤. 

2.过滤介质

过滤介质是滤饼的支承物,它应具有足够的力学强度和尽可能小的流动阻力,同时,还应具有相应的耐腐蚀性和耐热性. 

工业上常用的过滤介质主要有下面 3 类. 

(1)织物介质:(又称滤布)包括由棉,毛,丝,麻等天然纤维及合成纤维制成的织物,以及由玻璃丝,金属丝等织成的网. 这类介质能截留颗粒的最小直径为\(5 \sim 65\mu\text{\ }m\) . 织物介质在工业上应用最为广泛. 

(2)堆积介质:此类介质由各种固体颗粒(细砂,木炭,石棉,硅藻土)或非编织纤维等堆积而成,多用于深床过滤中. 

(3)多孔固体介质:这类介质是具有很多微细孔道的固体材料,如多孔陶瓷,多孔塑料及多孔金属制成的管或板,能拦截\(1 \sim 3\mu\text{\ }m\) 的微细颗粒. 

3.滤饼的压缩性和助滤剂

滤饼是由截留下的固体颗粒堆积而成的床层,随着操作的进行,滤饼的厚度与流动阻力都逐渐增加. 构成滤饼的颗粒特性对流动阻力的影响悬殊很大. 颗粒如果是不易变形的坚硬固体(如硅㩰土,碳酸钙等),则当滤饼两侧的压力差增大时,颗粒的形状和颗粒间的空際都不发生明亚变化,单位厚度床层的流动阻力可视为恒定,这类滤饼称为不可压缩滤饼. 相反,如果滤饼是由某些类似氢氧化物的胶体物质构成,则当滤饼两侧的压力差增大时,颗粒的形状和颗粒间的空隙便有明显的改变,单位厚度饼层的流动阻力随压力差增大而增大,这种滤饼称为可压缩滤饼. 

为了减小可压缩滤饼的流动阻力,有时将某种质地坚硬而能形成疏松饼层的另一种固体颗粒混入悬浮液或预涂于过滤介质上,以形成疏松饼层,使滤液得以畅流. 这种预混或预涂的粒状物质称为助滤剂. 

对助滤剂的基本要求:(1)应能形成多孔饼层的刚性颗粒,使滤饼有良好的渗透性,较高的空隙率及较小的流动阻力;(2)应具有化学稳定性,不与悬浮液发生化学反应,也不溶于液相中. 

应予注意,一般以获得清净滤液为目的时,采用助滤剂是适宜的. 

3.3.2 颗粒床层的特性及流体流过床层的压降

1.颗粒床层的特性

1)床层空隙率 \(\varepsilon\)

由颗粒群堆积成的床层疏密程度可用空隙率表示,其定义如下:

\[\varepsilon = \frac{\text{~床层体积~} - \text{~颗粒体积~}}{\text{~床层体积~}}\]

影响空隙率 \(\varepsilon\)值的因素非常复杂,诸如颗粒的大小,形状,粒度分布与充填方式等. 实验证明,单分散性球形颗粒作最松排列时的空隙率为0.48 ,作最紧密排列时为 0.26;乱堆的非球形颗粒床层空隙率往往大于球形的. 形状系数 \(\phi_s\),值愈小,空隙率 \(\varepsilon\) 值超过球形 \(\varepsilon\)的可能性愈大;多分散性颗粒所形成的床层空隙率则较小;若充填时设备受到振动,则空隙率必定小,采用湿法充填(即设备内先充以液体),则空隙率必大. 

一般乱堆床层的空隙率大致在 \(0.47 \sim 0.70\) 之间. 

在床层的同一截面上空隙率的分布通常是不均匀的. 容器壁面附近的空隙率大于床层中心的. 这种壁面的影响称为壁效应. 改善壁效应的方法是限制床层直径与颗粒定性尺寸之比不得小于某极限值. 若床层直径比颗粒尺寸大得多,则可忽略壁效应. 

2)床层的比表面积 \(a_{b}\)

单位床层体积具有的颗粒表面积称为床层的比表面积 \(a_{b}\). 若忽略颗粒之间接触面积的影响,则

\[a_{b} = (1 - \varepsilon)a\]

式中 \(a_{b}\) ---床层比表面积, \(m^{2}/m^{3}\) ;

\(a\) ------颗粒的比表面积, \(m^{2}/m^{3}\) ;

\(\varepsilon\) ------床层空隙率. 

床层比表面积也可根据堆积密度估算,即

\[a_{b} = \frac{6\rho_{b}}{d\rho_{s}}\]

式中 \(\rho_{b}\rho_{g}\) 分别为堆积密度和真实密度,\(kg/m^{3}. \rho_{b}\) 和 \(\rho_{s}\) ,之间的近似关系可用下式表示:

\[\rho_{b} = (1 - \varepsilon)\rho_{s}\]

3)床层的自由截面积

床层截面上未被颗粒占据的,流体可以自由通过的面积即为床层的自由截面积. 

工业上,小颗粒的床层用乱堆方法堆成,而非球形颗粒的定向是随机的,因而可认为床层是各向同性. 各向同性床层的一个重要特点是,床层横截面上可供流体通过的自由截面(即空隙截面)与床层截面之比在数值上等于空隙率\(\varepsilon\) . 

由于壁效应的影响,较多的流体必趋向近壁处流过,使床层截面上流体分布不均匀. 当床层直径\(D\) 与颗粒直径 \(d\) 之比 \(D/d\) 较小时,壁效应的影响尤为严重. 

2.流体通过床层流动的压降

固定床层中颗粒间的空隙形成可供流体通过的细小,曲折,互相交联的复杂通道. 流体通过如此复杂通道的流动阻力很难进行理论推算. 本节采用数学模型法进行研究. 

1)床层的简化模型

细小而密集的固体颗粒床层具有很大的比表面积,流体通过这样床层的流动多为滞流,流动阻力基本上为黏性摩擦阻力,从而使整个床层截面速度的分布均匀化. 为解决流体通过床层的压降计算问题,在保证单位床层体积表面积相等的前提下,将颗粒床层内实际流动过程加以简化,以便可以用数学方程式加以描述. 

简化模型是将床层中不规则的通道假设成长度为 \(L\) ,当量直径为 \(d_{e}\)
的一组平行细管,并且规定:(1)细管的全部流动空间等于颗粒床层的空隙容积;(2)细管的内表面积等于颗粒床层的全部表面积. 

在上述简化条件下,以 \(1{\text{\ }m}^{3}\)床层体积为基准,细管的当量直径可表示为床层空隙率\(\varepsilon\)及比表面积 \(a_{b}\) 的函数,即

\[d_{eb} = \frac{4 \times \text{~床层流动空间~}}{\text{~细管的全部内表面积~}} = \frac{4\varepsilon}{a_{b}} = \frac{4\varepsilon}{(1 - \varepsilon)a}\]

2)流体通过床层压降的数学描述

根据前述简化模型,流体通过一组平行细管流动的压降为

\[\Delta p_{f} = \lambda\frac{Lu_{1}^{2}}{\left( d_{eb} \right)2}\rho\]

式中 \(\Delta p_{f}\) ------流体通过床层的压降, Pa ;

\(L\) ------床层高度, m ;

\(d_{eb}\) ------床层流道的当量直径, m ;

\(u_{1}\) ------流体在床层内的实际流速, \(m/s\) . 

\(u_{1}\) 与按整个床层截面计算的空床流速 \(u\) 的关系为

\[u_{1} = \frac{u}{\varepsilon}\]

将式(3-32)与式(3-34)代人式(3-33),得到

\[\frac{\Delta p_{f}}{L} = \lambda^{'}\frac{(1 - \varepsilon)a}{\varepsilon^{3}}\rho u^{2}\]

式(3-35)即为流体通过固定床压降的数学模型,式中的 \(\lambda^{'}\)
为流体通过床层流道的摩擦系数,称为模型参数,其值由实验测定. 

3)模型参数的实验测定

模型的有效性需通过实验检验,模型参数需实验测定. 

(1)康采尼(Kozeny)实验结果 康采尼通过实验发现,在流速较低,床层雷诺数
\(Re_{b} < 2\)的层流情况下,模型参数 \(\lambda^{'}\) 可较好地符合下式:

\[\lambda^{'} = \frac{K^{'}}{Re_{b}}\]

式中 \(K^{'}\) 称为康采尼常数,其值可取 5.0 .  \(Re_{b}\) 的定义为

\[Re_{b} = \frac{d_{\text{eb}\text{~}}u_{1}\rho}{4\mu} = \frac{\rho u}{a(1 - \varepsilon)\mu}\]

式中 \(\mu\) ------流体的黏度, \(Pa \cdot s\) . 

将式(3-36)与式(3-37)代人式(3-35),即为康采尼方程式,即

\[\frac{\Delta p_{f}}{L} = 5\frac{(1 - \varepsilon)^{2}a^{2}u\mu}{\varepsilon^{3}}\]

(2)欧根(Ergun)实验结果 欧根在较宽的 \(Re_{b}\)
范围内进行实验,获得如下关联式

\[\lambda^{'} = \frac{4.17}{Re_{b}} + 0.29\]

将式(3-37),式(3-39)代人式(3-35),得到

\[\frac{\Delta p_{f}}{L} = 4.17\frac{(1 - \varepsilon)^{2}a^{2}u\mu}{\varepsilon^{3}} + 0.29\frac{(1 - \varepsilon)a\rho u^{2}}{\varepsilon^{3}}\]

将 \(a = 6/\left( \phi_{s}d_{e} \right)\) 代人上式,得到

\[\frac{\Delta p_{f}}{L} = 150\frac{(1 - \varepsilon)^{2}u\mu}{\varepsilon^{3}\left( \phi_{s}d_{e} \right)^{2}} + 1.74\frac{(1 - \varepsilon)\rho u^{2}}{\varepsilon^{3}\left( \phi_s d_{e} \right)}\]

式(3-41)称为欧根方程,适用于 \(Re_{b}\) 为 \(0.17 \sim 330\)
的范围. 当 \(Re_{b} < 20\)时,流动基本为层流,式(3-41)中等号右边第二项可忽略;当\(Re_{b} > 1000\) 时,流动为湍流,式(3-41)中等号右边第一项可忽略. 

3.3.3 过滤基本方程式

过滤基本方程式是描述过滤速率(或过滤速度)与过滤推动力,过滤面积,料浆性质,介质特性及滤饼厚度等诸因素关系的数学表达式. 本节从分析滤液通过滤饼层流动的特点人手,将复杂的实际流动加以简化,对滤液的流动用数学方程式进行描述,并以基本方程式为依据,分析强化过滤操作的途径,进行过滤计算. 

1.滤液通过滤饼层的流动

1)滤液通过滤饼层流动的特点

(1)滤液通道细小曲折,形成不规则的网状结构. 

(2)随着过滤进行,滤饼厚度不断增加,流动阻力逐渐加大,因而过滤属非稳态操作. 

(3)细小而密集的颗粒层提供了很大的液,固接触表面,滤液的流动大都在层流区. 

2)滤液通过滤饼层流动的数学描述

对于滤液通过平行细管的层流流动,由式(3-38)的康采尼方程式得到

\[u = \frac{\varepsilon^{3}}{5a^{2}(1 - \varepsilon)^{2}}\left( \frac{\Delta p_{c}}{\mu L} \right)\]

式中 \(u\) ------按整个床层截面积计算的滤液平均流速, \(m/s\) ;

\(\Delta p_{c}\) ------滤液通过滤饼层的压力降, Pa ;

\(L\) ------滤饼层厚度, m ;

\(\mu\) ------滤液秥度, \(Pa \cdot s\) . 

2.过滤速率和过滤速度

前面讨论的 \(u\)
为单位时间通过单位过滤面积的滤液体积,称为过滤速度. 通常将单位时间获得的滤液体积称为过滤速率,单位为\(m^{3}/s\). 过滤速度是单位过滤面积上的过滤速率,应防止将二者相混淆. 若过滤进程中其他因素维持不变,则由于滤饼厚度不断增加而使过滤速度逐渐变小. 任一瞬间的过滤速度应写成如下形式:

\[u = \frac{dV}{A\text{\ }d\theta} = \frac{\varepsilon^{3}}{5a^{2}(1 - \varepsilon)^{2}}\left( \frac{\Delta p_{c}}{\mu L} \right)\]

而过滤速率为

\[\frac{dV}{\text{\ }d\theta} = \frac{\varepsilon^{3}}{5a^{2}(1 - \varepsilon)^{2}}\left( \frac{A\Delta p_{c}}{\mu L} \right)\]

式中 \(V\) ------滤液量, \(m^{3}\) ;

\(\theta\)------过滤时间, s ;

\(A\) ------过滤面积, \(m^{2}\) . 

3.滤饼的阻力

对于不可压缩滤饼,滤饼层中的空隙率 \(\varepsilon\)可视为常数,颗粒的形状,尺寸也不改变,因而比表面 \(a\)亦为常数. 式(3-42a)和式(3-42b)中的\(\frac{\varepsilon^{3}}{5a^{2}(1 - \varepsilon)^{2}}\)反映了颗粒的特性,其值随物料而不同. 若以 \(r\)
代表其倒数,则式(3-42a)可写成

\[\begin{matrix}
 & \frac{dV}{A\text{\ }d\theta} = \frac{\Delta p_{c}}{\mu\text{\ }L} = \frac{\Delta p_{c}}{\mu R} \\
 & r = \frac{5a^{2}(1 - \varepsilon)^{2}}{\varepsilon^{3}} \\
 & R = rL \\
\end{matrix}\]

式中 \(r\) ------滤饼的比阻, \(1/m^{2}\) ;

\(R\) ------滤饼阻力, \(1/m\) . 

应指出,式(3-43)具有"速度 \(=\) 推动力/阻力"的形式,式中\(\mu rL\) 及 \(\mu R\) 均为过滤阻力. 显然 \(\mu r\) 为比阻,但因\(\mu\) 代表滤液的影响因素,\(rL\) 代表滤饼的影响因素,因此习惯上将\(r\) 称为滤饼的比阻,\(R\) 称为滤饼阻力. 

比阻 \(r\) 是单位厚度滤饼的阻力,它在数值上等于黏度为\(1\text{\ }Pa \cdot \text{\ }s\) 的滤液以 \(1\text{\ }m/s\)的平均流速通过厚度为 1 m的滤饼层时所产生的压力降. 比阻反映了颗粒形状,尺寸及床层空隙率对滤液流动的影响. 床层空隙率\(\varepsilon\) 愈小及颗粒比表面 \(a\)愈大,则床层愈致密,对流体流动的阻滞作用也愈大. 

4.过滤介质的阻力

饼层过滤中,过滤介质的阻力一般都比较小,但有时却不能忽略,尤其在过滤初始滤饼尚薄的期间. 过滤介质的阻力当然也与其厚度及本身的致密程度有关. 通常把过滤介质的阻力视为常数,仿照式(3-43)可以写出滤液穿过过滤介质层的速度关系式:

\[\frac{dV}{A\text{\ }d\theta} = \frac{\Delta p_{m}}{\mu R_{m}}\]

式中 \(\Delta p_{m}\) ------过滤介质上,下游两侧的压力差, Pa ;

\(R_{m}\) ------过滤介质阻力, \(1/m\) . 

由于很难划定过滤介质与滤饼之间的分界面,更难测定分界面处的压力,因而过滤介质的阻力与最初所形成的滤饼层的阻力往往无法分开,所以过滤操作中总是把过滤介质与滤饼联合起来考虑. 

通常,滤饼与滤布的面积相同,所以两层中的过滤速度应相等,则

\[\frac{dV}{A\text{\ }d\theta} = \frac{\Delta p_{c} + \Delta p_{m}}{\mu\left( R + R_{m} \right)} = \frac{\Delta p}{\mu\left( R + R_{m} \right)}\]

式中 \(\Delta p = \Delta p_{c} + \Delta p_{m}\)
,代表滤饼与滤布两侧的总压力降,称为过滤压力差. 在实际过滤设备上,常有一侧处于大气压下,此时\(\Delta p\) 就是另一侧表压的绝对值,所以 \(\Delta p\)也称为过滤的表压力. 式(3-47)表明,可用滤液通过串联的滤饼与滤布的总压力降来表示过滤推动力,用两层的阻力之和来表示总阻力. 

为方便起见,设想以一层厚度为 \(L_{e}\)的滤饼来代替滤布,而过程仍能完全按照原来的速率进行,那么,这层设想中的滤饼就应当具有与滤布相同的阻力,即

\[rL_{e} = R_{m}\]

于是,式(3-47)可写为

\[\frac{dV}{A\text{\ }d\theta} = \frac{\Delta p}{\mu\left( rL + rL_{e} \right)} = \frac{\Delta p}{\mu r\left( L + L_{e} \right)}\]

式中 \(L_{e}\) ------过滤介质的当量滤饼厚度,或称虚拟滤饼厚度,\(m_{\circ}\)

在一定的操作条件下,以一定介质过滤一定悬浮液时,\(L_{e}\)为定值;但同一介质在不同的过滤操作中,\(L_{e}\) 值不同. 

5.过滤基本方程式

若每获得 \(1{\text{\ }m}^{3}\) 滤液所形成的滤饼体积为\(v{\text{\ }m}^{3}\) ,则任一瞬间的汯饼厚度 \(L\)与当时已经获得的滤液体积 \(V\) 之间的关系应为

\[LA = vV\]

\[\text{~则~}\ L = \frac{vV}{A}\]

式中 \(v\) ---滤饼体积与相应的滤液体积之比,量纲为 1 ,或\(m^{3}/m^{3}\) . 

同理,如生成厚度为 \(L_{e}\) 的滤饼所应获得的滤液体积以 \(V_{e}\)
表示,则

\[L_{e} = \frac{vV_{e}}{A}\]

式中 \(V_{e}\) ------过滤介质的当量滤液体积,或称虚拟滤液体积,
\(m^{3}\) . 

在一定的操作条件下,以一定介质过滤一定的悬浮液时,\(V_{e}\)为定值,但同一介质在不同的过滤操作中,\(V_{e}\) 值不同. 

于是,式(3-48)可以写成:

\[\frac{dV}{A\text{\ }d\theta} = \frac{\Delta p}{\mu rv\left( \frac{V + V_{e}}{A} \right)}\]

\[\ \frac{dV}{\text{\ }d\theta} = \frac{A^{2}\Delta p}{\mu rv\left( V + V_{e} \right)}
\]

\[r = r^{'}(\Delta p)^{s}\]

式中 \(r^{'}\) ------单位压力差下滤饼的比阻, \(1/m^{2}\) ;

\(\Delta p\) ---过滤压力差, Pa ;

\(s\) ------滤饼的压缩性指数,量纲为 1 . 一般情况下,\(s = 0 \sim 1\)
,对于不可压缩滤饼,\(s = 0\). 几种典型物料的压缩性指数值,列于表3-5中. 

表3-5 典型物料的压缩性指数

\begin{longtable}[]{@{}
  >{\raggedright\arraybackslash}p{(\columnwidth - 16\tabcolsep) * \real{0.0864}}
  >{\raggedright\arraybackslash}p{(\columnwidth - 16\tabcolsep) * \real{0.1056}}
  >{\raggedright\arraybackslash}p{(\columnwidth - 16\tabcolsep) * \real{0.1056}}
  >{\raggedright\arraybackslash}p{(\columnwidth - 16\tabcolsep) * \real{0.1434}}
  >{\raggedright\arraybackslash}p{(\columnwidth - 16\tabcolsep) * \real{0.1056}}
  >{\raggedright\arraybackslash}p{(\columnwidth - 16\tabcolsep) * \real{0.0840}}
  >{\raggedright\arraybackslash}p{(\columnwidth - 16\tabcolsep) * \real{0.1369}}
  >{\raggedright\arraybackslash}p{(\columnwidth - 16\tabcolsep) * \real{0.1056}}
  >{\raggedright\arraybackslash}p{(\columnwidth - 16\tabcolsep) * \real{0.1272}}@{}}
\toprule()
\begin{minipage}[b]{\linewidth}\raggedright
物料
\end{minipage} & \begin{minipage}[b]{\linewidth}\raggedright
硅藻土
\end{minipage} & \begin{minipage}[b]{\linewidth}\raggedright
碳酸钙
\end{minipage} & \begin{minipage}[b]{\linewidth}\raggedright
钛白(絮凝)
\end{minipage} & \begin{minipage}[b]{\linewidth}\raggedright
高岭土
\end{minipage} & \begin{minipage}[b]{\linewidth}\raggedright
滑石
\end{minipage} & \begin{minipage}[b]{\linewidth}\raggedright
黏土
\end{minipage} & \begin{minipage}[b]{\linewidth}\raggedright
硫酸锌
\end{minipage} & \begin{minipage}[b]{\linewidth}\raggedright
氢氧化铝
\end{minipage} \\
\midrule()
\endhead
\(s\) & 0.01 & 0.19 & 0.27 & 0.33 & 0.51 & \(0.56 \sim 0.6\) & 0.69 &
0.9 \\
\bottomrule()
\end{longtable}

在一定的压力差范围内,上式对大多数可压缩滤饼都适用. 将式(3-52)代人式(3-51a),得到

\[\frac{dV}{\text{\ }d\theta} = \frac{A^{2}\Delta p^{1 - s}}{\mu r^{'}v\left( V + V_{e} \right)}\]

上式称为过滤基本方程式,表示过滤进程中任一瞬间的过滤速率与各有关因素间的关系,是过滤计算及强化过湃操作的基本依据. 该式适用于可压缩滤饼及不可压缩滤饼. 对于不可压缩滤饼,因\(s = 0\) ,上式即简化为式(3-51a). 

应用过滤基本方程式时,需针对操作的具体方式而积分. 过滤操作有两种典型的方式,即恒压过滤及恒速过滤. 有时,为避免过滤初期因压力差过高而引起滤液浑浊或滤布堵塞,可采用先恒速后恒压的复合操作方式,过滤开始时以较低的恒定速率操作,当表压升至给定数值后,再转人恒压操作. 当然,工业上也有既非恒速亦非恒压的过滤操作,如用离心䂞向压滤机送料浆即属此例. 

3.3.4 恒压过滤

若过滤操作是在恒定压力差下进行的,则称为恒压过滤. 恒压过滤是最常见的过滤方
式. 连续过滤机内进行的过滤都是恒压过滤,间歌过㴶机内进行的过滤也多为恒压过滤. 恒压过滤时滤饼不断变厚,致使阻力䦽渐增大,但推动力\(\Delta p\) 恒定,因面过滤速率逐䟟变小. 对于一定的隐浮液,若\(\mu,\text{\ }r^{'}\) 及 \(v\) 替可视为常数,令

\[k = \frac{1}{\mu r^{'}v}\]

式中 \(k\) ------表征过滤物料特性的常数, \(m^{4}/(N \cdot s)\) 或
\(m^{2}/(Pa \cdot s)\) . 

将式(3-54)代人式(3-53),得

\[\frac{dV}{\text{\ }d\theta} = \frac{kA^{2}\Delta p^{1 - s}}{V + V_{e}}\]

恒压过滤时,压力差 \(\Delta p\) 不变,\(k,\text{\ }A,\text{\ }s\)都是常数. 再令

\[K = 2k\Delta p^{1 - s}\]

将式(3-55)代人式(3-53a),得

\[\frac{dV}{\text{\ }d\theta} = \frac{KA^{2}}{2\left( V + V_{e} \right)}\]

对式(3-53b)积分,积分上下限为:过滤时间 \(0 \rightarrow \theta\),滤液体积 \(0 \rightarrow V\) ,即

\[\int_{0}^{V}\mspace{2mu}\left( V + V_{e} \right)dV = \frac{1}{2}KA^{2}\int_{0}^{\theta}\mspace{2mu} d\theta\]

得到

\[V^{2} + 2V_{e}V = KA^{2}\theta\]

若令 \(q = \frac{V}{A},q_{e} = \frac{V_{e}}{A}\) ,则式 \((3 - 56)\)变为

\[q^{2} + 2q_{e} \cdot q = K\theta\]

式(3-56)称为恒压过滤方程式,它表明恒压过滤时滤液体积与过滤时间的关系为拋物线方程. 

当过滤介质阻力可以忽略时,\(V_{e} = 0,q_{e} = 0\) ,则式(3-56)简化为

\[\begin{matrix}
 & V^{2} = KA^{2}\theta \\
 & q^{2} = K\theta \\
\end{matrix}\]

式(3-57)也称为恒压过滤方程式. 

恒压过滤方程式中的 \(K\)是由物料特性及过涏压力差所决定的常数,称为过滤常数,其单位为\(m^{2}/s;V_{e}\) 与 \(q\). 是反映过滤介质䧋力大小的常数,均称为介质常数,其单位分别为 \(m^{3}\)及\(m^{3}/m^{2}\) ,三者总称过滤常数,其数值由实验测定. 

3.3.5 恒速过滤与先恒速后恒压过滤

过滤设备(如板框压滤机)内部空间的容积是一定的,当料浆充满此空间后,供料的体积流量就等于滤液流出的体积流量,即过滤速率. 所以,当用排量固定的正位移泵向过滤机供料而未打开支路阀时,过滤速率便是恒定的. 这种维持速率恒定的过滤方式称为恒速过滤. 

恒速过滤时的过滤速度为

\[\frac{dV}{A\text{\ }d\theta} = \frac{V}{A\theta} = \frac{q}{\theta} = u_{R} = \text{~常数~}\]

所以

\[q = u_{R}\theta\]

\[V = Au_{\text{R}\text{~}}\theta
\]

\[\frac{dq}{\text{\ }d\theta} = \frac{\Delta p}{\mu rv\left( q + q_{e} \right)} = u_{R} = \text{~常数~}\]

在一定的条件下,式中的 \(\mu,\text{\ }r,\text{\ }v,\text{\ }u_{R}\) 及
\(q\) . 均为常数,仅 \(\Delta p\) 及 \(q\) 随 \(\theta\)而变化,于是得到

\[\Delta p = \mu rvu_{R}^{2}\theta + \mu rvu_{R}q_{e}\]

或写成 \(\Delta p = a\theta + b\)\\
式中常数:\(\ a = \mu rvu_{R}\ ^{2},b = \mu rvu_{R}q\) . 

式(3-60a)表明,对不可压缩滤饼进行恒速过滤时,其操作压力差随过滤时间成直线增高. 所以,实际上很少采用把恒速过滤进行到底的操作方法,而是采用先恒速后恒压的复合式操作方法. 这种复合式的装置见图3-16. 

由于采用正位移泵,过滤初期维持恒定速率,泵出口表压强逐渐升高. 经过\(\theta_{R}\) 时间后,获得体积为 \(V_{R}\)的滤液,若此时表压恰已升至能使支路阀自动开启的给定数值,则开始有部分料浆返回泵的人口,进人压滤机的料浆流量逐渐减小,而压滤机入口表压维持恒定. 后阶段的操作即为恒压过滤. 

对于恒压阶段的\(V - \theta\)关系,仍可用过滤基本方程式(3-53a)求得,即

\[\text{~或~}\ \left( V + V_{e} \right)dV = kA^{2}\Delta p^{1 - s}\text{\ }d\theta\]

若令 \(V_{R},\text{\ }\theta_{R}\)分别代表升压阶段终了瞬间的滤液体积及过滤时间,则上式的积分形式为

\[\int_{V_{n}}^{V}\mspace{2mu}\left( V + V_{e} \right)dV = kA^{2}\Delta p^{1 - s}\int_{\theta_{R}}^{\theta}\mspace{2mu} d\theta\]

积分上式并将式(3-55)代人,得

\[\left( V^{2} - V_{R}\ ^{2} \right) + 2V_{e}\left( V - V_{R} \right) = KA^{2}\left( \theta - \theta_{R} \right)\]

此式即为恒压阶段的过滤方程,式中\(V - V_{R},\text{\ }\theta - \theta_{R}\)分别代表转人恒压操作后所获得的滤液体积及所经历的过滤时间. 

3.3.6 过滤常数的测定

1.恒压下 \(K,\text{\ }V_{e}\left( q_{e} \right)\) 的测定\\在某指定的压力差下对一定料浆进行恒压过滤时,式(3-56)中的过滤常数\(K,\text{\ }V_{e}\left( q_{e} \right)\) 可通过恒压过滤实验测定. 

将恒压过滤方程式(3-56a)变换为

\[\frac{\theta}{q} = \frac{1}{K}q + \frac{2}{K}q_{e}\]

上式表明 \(\frac{\theta}{q}\) 与 \(q\) 呈直线关系,直线的斜率为\(\frac{1}{K}\) ,截距为 \(\frac{2}{K}q_{e}\) .

在过滤面积 \(A\) 上对待测的是浮料浆进行恒压过滤试验,测出一系列的时刻\(\theta\) 上的累积滤液量 \(V\) ,并由此算出一系列\(q\left( = \frac{V}{A} \right)\) 值. 在直角坐标系中标绘\(\frac{\theta}{q}\) 与 \(q\) 间的函数关系,可得一条直线. 由直线的斜率\(\left( \frac{1}{K} \right)\) 及截距\(\left( \frac{2}{K}q_{*} \right)\) 的数值即可求得 \(K\) 与 \(q_{e}\),再用 \(V_{e} = q_{e}A\) 即可求出 \(V_{\text{e}\text{~}\text{\ }}\)这样得到的 \(K,\text{\ }V_{e}\left( q_{e} \right)\)便是此种悬浮料浆在特定的过滤介质及压力差条件下的过滤常数. 

在过滤实验条件比较困难的情况下,只要能㓩获得指定条件下的过滤时间与滤液量的两组对应数据,也可计算出3 个过滤常数,因为

\[q^{2} + 2q_{e}q = K\theta\]

此式中只有 \(K,\text{\ }q\) ,两个未知量. 将已知的两组 \(q - \theta\)
对应数据代人该式,便可解出 \(q\), 及\(K\). 但是,如此求得的过滤常数,其准确性完全依赖于这仅有的两组数据,可靠程度往往较差. 

2.压缩性指数 \(s\) 的测定

为了进一步求得滤饼的压缩性指数 \(s\) 以及物料特性常数 \(k\),需要先在若干不同的压力原下对指定物料进行实验,求得若干过滤压力差下的\(K\) 值,然后对 \(K——\Delta p\) 数据加以处理,即可求得 s 值. 

a\[K = 2k\Delta p^{1 - s}\]

上式两端取对数,得

\[lgK = (1 - s)lg(\Delta p) + lg(2k)\]

因 \(k = \frac{1}{\mu r^{'}v} =\) 常数,故 \(K\) 与 \(\Delta p\)的关系在对数坐标纸上标绘时应是直线,直线的斜率为 \(1 - s\) ,截距为\(lg(2k)\) . 如此可得滤饼的压缩性指数 \(s\) 及物料特性常数 \(k\) . 

值得注意的是,上述求压缩性指数的方法是建立在 \(v\)值恒定的条件上的,这就要求在过滤压力变化范围内,滤饼的空隙率应没有显著的改变. 

3.3.7 过滤设备

各种生产工艺的悬浮液,其性质有很大的差异;过滤的目的及料浆的处理量相差也很悬殊. 为适应各种不同的要求而发展了多种形式的过滤机. 按照操作方式可分为间歇过滤机与连续过滤机;按照采用的压差可分为压滤、吸滤和离心过滤机. 工业上应用最广泛的板框压滤机和加压叶滤机为间歇压滤型过滤机,转筒真空过滤机则为吸滤型连续过滤机. 以下是对板框过滤机的简介.

1.板框压滤机

板框压滤机早为工业所使用,至今仍沿用不衰. 它由多块带凹凸纹路的滤板和滤框交替排列组装于机架上而构成,如图3-17所示. 

板和框一般制成正方形,如图3-18所示. 板和框的角端均开有圆孔,装合、压紧后即构成供滤浆、滤液或洗涤液流动的通道. 框的两侧覆以四角开孔的滤布,空框与滤布围成了容纳滤浆及滤饼的空间. 滤板又分为洗涤板与过滤板两种. 洗涤板左上角的圆孔内还开有与板面两侧相通的侧孔道,洗水可由此进人框内. 为了便于区别,常在板、框外侧铸有小钮或其他标志,通常,过滤板为一钮,洗涤板为三钮,而框则为二钮(如图3-18所示). 装合时即按钮数以1-2-3-2-1-2\ldots\ldots 的顺序排列板与框. 压紧装置的驱动可用手动、电动或液压传动等方式. 

过滤时,悬浮液在指定的压力下经滤浆通道由滤框角端的暗孔进人框内,滤液分别穿过两侧滤布,再经邻板板面流至滤液出口排走,固体则被截留于框内,如图3-19(a)所示,待滤饼充满滤框后,即停止过滤. 滤液的排出方式有明流与暗流之分. 若滤液经由每块滤板底部侧管直接排出(如图3-19所示),则称为明流. 若滤液不宜暴露于空气中,则需将各板流出的滤液汇集于总管后送走(如图3-17所示),称为暗流. 

若滤饼需要洗涤,可将洗水压人洗水通道,经洗涤板角端的暗孔进人板面与滤布之间此时,应关闭洗涤板下部的滤液出口,洗水便在压强差推动下穿过一层滤布及整个厚度的滤饼,然后再横穿另一层滤布,最后由过滤板下部的滤液出口排出,如图2-19(b)所示. 这种操作方式称为横穿洗涤法,其作用在于提高洗涤效果. 

洗涤结束后,旋开压紧装置并将板框拉开,卸出滤饼,清洗滤布,重新装合,进入下一个操作循环. 

板框压滤机的操作表压,一般在3x10'\textasciitilde8x10'Pa的范围内,有时可高达15x10\textquotesingle Pa. 滤板和滤框可由多种金属材料(如铸铁、碳钢、不锈钢、铝等)塑料及木材制造. 我国编制的压滤机系列标准及规定代号,如BMY50/810-25,其中,B表示板框压滤机,M表示明流式(若为A,则表示暗流式),Y表示油压压紧(若为S,则表示手动压紧),50表示过滤面积为50m\textquotesingle,810表示框内每边长810mm,25表示滤框厚度为25mm. 框每边长为320\textasciitilde1000mm,厚度为
25\textasciitilde50mm. 滤板和滤框的数目,可根据生产任务自行调节,一般为10\textasciitilde60块,所提供的过滤面积为2\textasciitilde80m. 当生产能力小、所需过滤面积较少时,可于板框间插入一块盲板,以切断过滤通道,盲板后部即失去作用. 

板框压滤机结构简单、制造方便、占地面积较小而过滤面积较大,操作压力高,适应能力强,故应用颇为广泛. 它的主要缺点是间歇操作,生产效率低,劳动强度大,滤布损耗也较快. 近来,各种自动操作板框压滤机的出现,使上述缺点在一定程度上得到改善. 

3.3.8 滤饼的洗涤

洗涤滤饼的目的在于回收残留在颗粒缝隙间的滤液,或净化构成滤饼的颗粒. 单位时间内消耗的洗水体积称为洗涤速率,以\(\left( \frac{dV}{\text{\ }d\theta} \right)_{W}\)表示. 由于洗水里不含固相,洗涤过程中滤饼厚度不变,因而,在恒定的压强差推动下洗涤速率基本为常数. 若每次过滤终了以体积为\(V_{w}\) 的洗水洗涤滤饼,则所需洗涤时间为

\[\theta_{w} = \frac{V_{w}}{\left( \frac{\text{\ }dV}{\text{\ }d\theta} \right)_{w}}\]

式中 \(V_{w}\) ---洗水用是, \(m^{3}\) ;

\(\theta_{w}\) ------洗涤时间, s . 

影响洗涤速率的因素可根据过滤基本方程式来分析,即

\[\frac{dV}{\text{\ }d\theta} = \frac{A\Delta p^{1 - s}}{\mu r^{'}\left( L + L_{e} \right)}\]

对于一定的悬浮液,\(r^{'}\)为常数. 若洗涤推动力与过滤终了时的压力差相同,并假设洗水关系取决于过滤设备采用的洗涤方式. 

叶滤机等所采用的是置换洗涤法,洗水与过滤终了时的滤液流过的路径基本相同,故

\[\left( L + L_{e} \right)_{W} = \left( L + L_{e} \right)_{E}\]

(式中下标 E表示过滤终了时刻)面且洗涤面积与过滤面积也相同,故洗涤速率大致等于过滤终了时的过滤速率,即

\[\left( \frac{dV}{\text{\ }d\theta} \right)_{w} = \left( \frac{dV}{\text{\ }d\theta} \right)_{E} = \frac{KA^{2}}{2\left( V + V_{e} \right)}\]

式中 \(V\) ------过滤终了时所得浗液体积, \(m^{3}\) . 

板框压滤机采用的是横穿洗涤法,洗水横穿两层滤布及整个厚度的滤饼,流径长度约为过滤终了时滤液流动路径的两倍,而供洗水流通的面积又仅为过滤面积的一半,即

\[\begin{matrix}
 & \left( L + L_{e} \right)_{w} = 2\left( L + L_{e} \right)_{E} \\
 & A_{v} = \frac{1}{2}A \\
\end{matrix}\]

将以上关系代入过滤基本方程式,可得

\[\left( \frac{dV}{\text{\ }d\theta} \right)_{V} = \frac{1}{4}\left( \frac{\text{\ }dV}{\text{\ }d\theta} \right)_{E} = \frac{KA^{2}}{8\left( V + V_{e} \right)}\]

即板框压滤机上的洗涤速率约为过滤终了时过滤速率的四分之一. 

当洗水黏度、洗水表压与滤液黏度、过滤压力差有明显差异时,所需的洗涤时间可按下式校正,即

\[\theta_{W}^{'} = \theta_{W}\left( \frac{\mu_{W}}{\mu} \right)\left( \frac{\Delta p}{\Delta p_{W}} \right)\]

式中 \(\theta_{W}\ ^{'}\) ------校正后的洗涤时间, s ;

\(\theta \approx\) ------未经校正的洗涤时间, s ;

\(\mu_{W}\) ------洗水黏度, \(Pa \cdot s\) ;

\(\Delta p\) ------过滤终了时刻的推动力, Pa ;

\(\Delta p_{w}\) ------洗涤推动力, Pa . 

3.3.9 过滤机的生产能力

过滤机的生产能力通常是指单位时间获得的滤液体积,少数情况下也有按滤饼的产量或滤饼中固相物质的产量来计算的. 

1.间歇过滤机的特点是在整个过滤机上依次进行过滤、洗涤、卸渣、清理、装合等步骤的循环操作. 在每一循环周期中,全部过滤面积只有部分时间在进行过滤,而过滤之外的各步操作所占用的时间也必须计人生产时间内. 因此在计算生产能力时,应以整个操作周期为基准. 操作周期为

\[T = \theta + \theta_{w} + \theta_{v}\]

式中 \(T\) ---个探作循环的时间,即操作周期, s ;

\(\theta -\) 一个操作循环内的过滤时间, s ;

\(\theta_{\mathbf{W}}\) ------一个操作循环内的洗涤时间, s ;

\(\theta_{0}\)
---------个操作循环内的卸渣,清理,装合等辅助操作所需时间,\(s\) . 

则生产能力的计算式为

\[Q = \frac{3600\text{\ }V}{T} = \frac{3600\text{\ }V}{\theta + \theta_{W} + \theta_{0}}\]

式中 \(V ~——\) 一个操作循环内所获得的滤液休积, \(m^{3}\);

\(Q\text{~——生产}\text{能}\text{力}\text{,}\text{~}m^{3}/h\text{~. ~}\)

【例3-11】对例3-10 中的悬浮流用有 26 个框的BMS20/635-25
板框压滤机进行过滤。在过滤机入口处滤浆的表压为\(3.39 \times 10^{5}\text{\ }Pa\) ,所用滤布与实验时的相同,浆料温度仍为\(25^{\circ}C\)。每次过滤完毕用清水洗涤滤饼,洗水温度及表压与滤浆相同面洗水体积为滤液体积的\(8\%\) 。每次卸渣,清理,装合等辅助操作时间为 15 min 。已知固相密度为\(2930\text{\ }kg/m^{3}\) ,又测得湿饼密度为 \(1930\text{\ }kg/m^{3}\)。求此板框压滤机的生产能力。

解:过滤面积\(A = (0.635)^{2} \times 2 \times 26 = 21{\text{\ }m}^{2}\)

滤框总容积\(= (0.635)^{2} \times 0.025 \times 26 = 0.262{\text{\ }m}^{3}\)

已知 \(1{\text{\ }m}^{3}\) 滤饼的质量为 1930 kg ,设其中含水\(x\text{\ }kg\) ,水的密度按 \(1000\text{\ }kg/m^{3}\) 计,则

\[\frac{1930 - x}{2930} + \frac{x}{1000} = 1\]

解得 \(x = 518\text{\ }kg\)

故知 \(1{\text{\ }m}^{3}\) 滤饼中的固相质量为\(1930 - 518 = 1412\text{\ }kg\) 。

生成 \(1{\text{\ }m}^{3}\) 滤饼所需的滤浆质量为\[1412 \times \frac{1000 + 25}{25} = 57890\text{\ }kg\]

则 \(1{\text{\ }m}^{3}\) 滤饼所对应的滤液质量为\(57890 - 1930 = 55960\text{\ }kg\) 。

\(1{\text{\ }m}^{3}\) 滤饼所对应的滤液体积为\[\frac{55960}{1000} = 55.96{\text{\ }m}^{3}\]

由此可知,滤框全部充满滤饼时的滤液体积为\[V = 55.96 \times 0.262 = 14.66{\text{\ }m}^{3}\]

则过滤终了时的单位面积滤液量为\[q = \frac{V}{A} = \frac{14.66}{21} = 0.6981{\text{\ }m}^{3}/m^{2}\]

根据例 3-10 中过滤实验结果写出\(\Delta p = 3.39 \times 10^{5}\text{\ }Pa\) 时的恒压过滤方程式为\[q^{2} + 0.0458q = 1.682 \times 10^{- 4}\theta\]

将 \(q = 0.6981{\text{\ }m}^{3}/m^{2}\) 代人上式,得\[{0.6981}^{2} + 0.0458 \times 0.6981 = 1.682 \times 10^{- 4}\theta\]

解得过滤时间为 \(\theta = 3087\text{\ }s\) 。

由式(3-62)及式(3-64)可知:\(\theta_{w} = \frac{V_{w}}{\frac{1}{4}\left( \frac{\text{\ }dV}{\text{\ }d\theta} \right)}\)

由恒压过滤方程式(3-53b)得

\[\frac{dq}{\text{\ }d\theta} = \frac{K}{2\left( q + q_{e} \right)}\]

已求得过滤终了时 \(q = 0.6981{\text{\ }m}^{3}/m^{2}\),代人上式可得过沷终了时的过滤速率为

\[\left( \frac{dV}{\text{\ }d\theta} \right)_{E} = A\frac{K}{2\left( q + q_{e} \right)} = 21 \times \frac{1.682 \times 10^{- 4}}{2(0.6981 + 0.0229)} = 2.450 \times 10^{- 3}{\text{\ }m}^{3}/s\]

已知

\[V_{w} = 0.08\text{\ }V = 0.08 \times 14.66 = 1.173{\text{\ }m}^{3}\]

则

\[\theta_{w} = \frac{1.173}{\frac{1}{4}\left( 2.450 \times 10^{- 3} \right)} = 1915\text{\ }s\]

又知 \(\theta_{D} = 15 \times 60 = 900\text{\ }s\)

则生产能力为

\[Q = \frac{3600\text{\ }V}{T} = \frac{3600\text{\ }V}{\theta + \theta_{w} + \theta_{v}} = \frac{3600 \times 14.66}{3087 + 1915 + 900} = 8.942{\text{\ }m}^{3}/h\]

\end{document}