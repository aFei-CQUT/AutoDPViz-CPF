% Options for packages loaded elsewhere
\PassOptionsToPackage{unicode}{hyperref}
\PassOptionsToPackage{hyphens}{url}
%
\documentclass[
]{article}
\usepackage{amsmath,amssymb}
\usepackage{lmodern}
\usepackage{iftex}
\ifPDFTeX
  \usepackage[T1]{fontenc}
  \usepackage[utf8]{inputenc}
  \usepackage{textcomp} % provide euro and other symbols
\else % if luatex or xetex
  \usepackage{unicode-math}
  \defaultfontfeatures{Scale=MatchLowercase}
  \defaultfontfeatures[\rmfamily]{Ligatures=TeX,Scale=1}
\fi
% Use upquote if available, for straight quotes in verbatim environments
\IfFileExists{upquote.sty}{\usepackage{upquote}}{}
\IfFileExists{microtype.sty}{% use microtype if available
  \usepackage[]{microtype}
  \UseMicrotypeSet[protrusion]{basicmath} % disable protrusion for tt fonts
}{}
\makeatletter
\@ifundefined{KOMAClassName}{% if non-KOMA class
  \IfFileExists{parskip.sty}{%
    \usepackage{parskip}
  }{% else
    \setlength{\parindent}{0pt}
    \setlength{\parskip}{6pt plus 2pt minus 1pt}}
}{% if KOMA class
  \KOMAoptions{parskip=half}}
\makeatother
\usepackage{xcolor}
\usepackage{longtable,booktabs,array}
\usepackage{multirow}
\usepackage{calc} % for calculating minipage widths
% Correct order of tables after \paragraph or \subparagraph
\usepackage{etoolbox}
\makeatletter
\patchcmd\longtable{\par}{\if@noskipsec\mbox{}\fi\par}{}{}
\makeatother
% Allow footnotes in longtable head/foot
\IfFileExists{footnotehyper.sty}{\usepackage{footnotehyper}}{\usepackage{footnote}}
\makesavenoteenv{longtable}
\setlength{\emergencystretch}{3em} % prevent overfull lines
\providecommand{\tightlist}{%
  \setlength{\itemsep}{0pt}\setlength{\parskip}{0pt}}
\setcounter{secnumdepth}{-\maxdimen} % remove section numbering
\ifLuaTeX
  \usepackage{selnolig}  % disable illegal ligatures
\fi
\IfFileExists{bookmark.sty}{\usepackage{bookmark}}{\usepackage{hyperref}}
\IfFileExists{xurl.sty}{\usepackage{xurl}}{} % add URL line breaks if available
\urlstyle{same} % disable monospaced font for URLs
\hypersetup{
  hidelinks,
  pdfcreator={LaTeX via pandoc}}

% 添加中文支持
\usepackage[UTF8, fontset=windows]{ctex}  % 启用中文支持,并使用 Windows 系统的默认字体

% 设置西文字体(如果需要的话,可以修改为其他字体)
\setmainfont{Times New Roman}  % 设置英文正文字体为 Times New Roman

% 设置数学字体
\setmathfont{Latin Modern Math}  % 设置数学字体

\author{}
\date{}

\begin{document}

\section{实验8 恒压过滤常数测定}

\subsection{一,实验目的}

(1)熟悉板框压滤机的构造和操作方法。

(2)掌握恒压过滤常数 $K, ~ q_e, ~ \theta_e$ 的测定方法,加深对 $K, ~ q_e, ~ \theta_e$ 概念和影响因素的理解。

(3)了解过滤压力对过滤速率的影响,学习滤饼的压缩性指数 和物料特性常数 $k$ 的测定方法。

(4)学习 $\frac{d \theta}{d q}-q$ 一类关系的实验确定方法。

(5)测定洗涤滤饼的洗涤速率,验证过滤终了速率和洗涤速率的关系。

\subsection{二,实验内容}

(1)测定不同压力下的过滤常数 $K, ~ q_e, ~ \theta_e$ 。

(2)根据实验测量数据,计算滤饼的压缩性指数 $s$ 和物料特性常数 $k_{\circ}$

\subsection{三,实验原理及过滤常数的求取}

1.实验原理

过滤是利用过滤介质进行液-固系统的分离过程,过滤介质通常采用带有许多毛细孔的物质如帆布,毛毯,多孔陶瓷等。含有固体颗粒的悬浮液在一定压力作用下,液体通过过滤介质,固体颗粒被截留,从而使液固两相分离。

在过滤过程中,由于固体颗粒不断地被截留在介质表面上,滤饼厚度逐渐增加,使得液体流过固体颗粒之间的孔道加长,增加了流体流动阻力。故恒压过滤时,过滤速率是逐渐下降的。随着过滤的进行,若想得到相同的滤液量,则过滤时间要增加。

2.$K, ~ q_e$ 和 $\theta_e$ 的求取

对恒压过滤,有:

$$
\left(q+q_e\right)^2=K\left(\theta+\theta_e\right)
$$

式中 $q$ —单位过滤面积获得的滤液体积, $m^3 / m^2$ ;
$q_e$ ——单位过滤面积上的虚拟滤液体积, $m^3 / m^2$ ;
$\theta$ ——实际过滤时间, $s$ ;
$\theta_e$ —虚拟过滤时间, $s$ ;
$K$ ——过滤常数, $m^2 / s$ 。
将式(3-31)进行微分可得:

$$
\frac{d \theta}{d q}=\frac{2}{K} q + \frac{2}{K} q_e
$$

这是一个直线方程式,由于实验过程中不可能测量到无穷小时间段内的滤液体积的变化,只能测得有限时间段内的滤液体积,当各数据点的时间间隔不大时,$\frac{d \theta}{d q}$ 可用增量之比 $\frac{\Delta \theta}{\Delta q}$ 来代替于普通坐标上标绘 $\frac{d \theta}{d q}-q$ 的关系曲线:

$$
\frac{\Delta \theta}{\Delta q}=\frac{2}{K} \bar{q} + \frac{2}{K} q_e
$$

式中 $\Delta q$ —每次测定的单位过滤面积滤液体积(实验中等量分配) $m^3 / m^2$ ;
$\Delta \theta$ —每次测定的滤液体积 $\Delta q$ 所对应的过滤时间,$s$ ;
$\bar{q}$ ——相邻两个 $q$ 值的平均值, $m^3 / m^2$ 。
注:式(3-33)中 $\bar{q}$ 代替式(3-32)的 $q$ ,原因可参见化工原理教材中过滤常数测定章节。

在直角坐标系中,以 $\frac{\Delta \theta}{\Delta q}$ 为纵坐标,相对应的 $\bar{q}$ 为横坐标绘图,可得一直线,直线的斜率为 $\frac{2}{K}$ ,截距 为 $\frac{2}{K} q_e$ ,从而求出 $K, ~ q_e$ 。至于 $\theta_e$ 可由式(3-34)求出:

$$
q_e^2=K \theta_e
$$

改变过滤压力,可得到不同操作压力下的过滤常数 $K$ 值,根据过滤常数的定义式:

$$
K=2 k \Delta p^{1-s}
$$

两边取对数:

$$
\lg K=(1-s) \lg \Delta p + \lg (2 k)
$$

在不同压力过滤时,因 $k=\frac{1}{\mu r^{\prime} v}=$ 常数 ,故 $K$ 与 $\Delta p$ 的关系在对数坐标上标绘时应是一条直线,直线的斜率为 $1-s$ ,由此可得滤饼的压缩性指数 $s$ ,然后代入式(3-35)求物料特性常数 $k_{\circ}$ 。

\subsection{四,实验装置和流程}

2.实验装置二

(1)实验装置

实验装置流程如图3-25所示。原料槽内配有一定浓度的轻质碳酸钙悬浮液(浓度在 $6\%$ $\sim 8\%$ 左右),用电动搅拌器进行均匀搅拌(以浆液不出现旋涡为好)。启动旋涡泵 19,调节阀门7使压力表11指示在规定值。滤液在计量槽内计量。

1—搅拌电机;2—原料罐;3—搅拌挡板;4,14,15—排液阀;5—洗水槽;6,7—调节阀;8—温度计;9,10,12—阀门;11—压力表;13—滤液计量槽;16—板框压滤机;17—过滤机压紧装置;18—过滤板;19—旋涡泵实验装置中过滤、洗涤管路分布如图3-26所示。

(2)实验设备主要技术参数(表3-30)
\begin{table}[h]
    \centering
    \caption{表3-30 恒压过滤常数测定实验设备主要技术参数}
    \begin{tabular}{c|c|c|c}
    \hline 序号 & 名称 & 规格 & 材料 \\
    \hline 1 & 搅拌器 & 型号:KDZ-1 & \\
    \hline 2 & 过滤板 & $160 mm \times 180 mm \times 11 mm$ & 不锈钢 \\
    \hline 3 & 滤布 & 工业用 & \\
    \hline 4 & 过滤面积 & $0.0475 m^2$ & \\
    \hline 5 & 计量桶 & 长 327 mm ,宽 286 mm & \\
    \hline
    \end{tabular}
\end{table}

(3)实验装置面板图(图3-27)

图3-27 恒压过滤常数测定实验设备仪表面板示意图

\subsection{五,实验操作及注意事项}

2.实验装置二

(1)实验步骤

1)实验准备工作

(1)原料罐内配好浓度在 $6\% \sim 8\%$ 左右的轻质碳酸钙悬浮液,系统接上电源,开启总电源,开启搅拌,使料浆搅拌均匀。

(2)在滤液水槽中加入一定高度液位的水(水位在标尺 50 mm 处即可)。

(3)板框过滤机板,框排列顺序为固定头 $\rightarrow$ 非洗涤板 $(\cdot) 
\rightarrow$ 框 $(:) \rightarrow$ 洗涤板 $(⋮)\rightarrow$ 框 $(:) \rightarrow$ 非洗涤板$(\cdot)$$\rightarrow$ 可动头。用压紧装置压紧后待用。

2)过滤实验

(1)阀门9,7全开,其他阀门全部关闭(图3-23)。启动旋涡泵19,打开阀门12,利用料液回水阀7调节压力,使压力表11达到规定值。

(2)待压力表11数值稳定后,打开过滤机滤液入口阀A,随后快速打开过滤机出口阀门B,C开始过滤。当计量桶13内见到第一滴液体时开始计时,记录滤液每增加高度 10 mm 时所用的时间。当计量桶 13 读数为 150 mm 时停止计时,并立即关闭过滤机进料阀 B。

(3)打开料液回水阀 7 使压力表11指示值下降,关闭泵开关。放出计量槽内的滤液倒回槽内,以保证料浆浓度保持不变。

3)洗涤实验

(1)洗涤实验时全开阀门10,6,其他阀门全关。调节阀门6使压力表11达到过滤要求的数值。打开阀门B,随后快速打开过滤机出口阀门C开始洗涤。等到阀门B有液体流下时开始计时,洗涤量为过滤量的 $1 / 4$ 。实验结束后,放出计量槽内的滤液到洗水槽 5 内。

(2)开启压紧装置,卸下过滤框内的滤饼并放回滤浆槽内,将滤布清洗干净。

(3)改变压力值,从开始重复上述实验。压力分别为 $0.05 MPa, ~ 0.10 MPa, \\~ 0.15 MPa$ 。

(2)操作注意事项

1)过滤板与过滤框之间的密封垫注意要放正,过滤板与过滤框上面的滤液进出口要对齐。滤板与滤框安装完毕后要用摇柄把过滤设备压紧,以免漏液。

2)计量槽的流液管口应紧贴桶壁,防止液面波动影响读数。

3)由于电动搅拌器为无级调速,使用时首先接上系统电源,打开调速器开关,调速钮一定由小到大缓慢调节,切勿反方向调节或调节过快,以免损坏电机。

4)启动搅拌前,用手旋转一下搅拌轴以保证启动顺利。

5)每次实验结束后将滤饼和滤液全部倒回料浆槽中,保证料液浓度保持不变。

\end{document}