% ====================== 结语 ======================

\section{结语}

本数字化设计通过自动化技术和先进数据处理方法,显著优化了恒压过滤实验的教学流程。系统利用图像识别技术实时监测滤液液位,并结合机器学习算法分析数据,不仅减少了人工操作时间,还提升了实验效率与结果准确性。该设计在教学中提供了数据驱动的学习体验,帮助学生掌握现代化工领域的核心数字化技能,具有一定的教育意义。其创新性体现在图像识别、机器学习与集成平台的结合(目前已包含7个实验),简化操作并提高精度。未来,该系统可扩展至更多实验场景,推动化学工程教育的数字化转型。