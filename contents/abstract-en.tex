% ====================== 英文摘要 ======================

% 英文摘要
\noindent\textbf{Abstract:}\abstracten{To address the challenges of time-consuming manual data recording and cumbersome processing in traditional constant-pressure filtration experiments of chemical engineering principles, this study proposes a smart experimental platform integrating hardware and software with digitalization as its core. Leveraging digital image recognition technology (OpenCV), the system achieves automated calibration of the filtrate tank, real-time liquid level monitoring, and data storage. A digital experimental data processing system (ChemLabX), developed in Python, dynamically filters data subsets according to the requirements of constant-pressure filtration experiments. The differential method is employed to compute the $\Delta\theta/\Delta q$ sequence, while a Z-score outlier detection mechanism dynamically eliminates anomalies. Furthermore, a linear regression model based on the Scikit-learn machine learning framework is established to digitally refit the filtration equation, thereby enhancing model accuracy. The processed results and visualizations are integrated into a user interface. By digitally reconstructing the experimental workflow, this solution significantly streamlines traditional manual operations, demonstrating advantages of high precision, efficiency, and reproducibility. It provides a standardized and intelligent reference framework for educational, research, and industrial applications in chemical engineering.}

% 竖直间距
\vspace{0.5em}

% 英文关键词
\noindent\textbf{keywords:}\keywordsen{Constant pressure filtration experiment; Image recognition; Automated data collection and processing; Data visualization; Python software development}
