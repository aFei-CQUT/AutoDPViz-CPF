\section{数字化设计方案}

% 请写明体现实验本身的教学目标及通过数字化设计所能够实现的教学目标和教学功能;数字化设计所针对的原型实验的原理和内容;数字化设计内容的科学原型和原理、数学物理建模过程、数据库及算法等。

化工原理恒压过滤实验要求学生熟悉相应的操作器械和操作方法,通过对理论知识的学习加深对恒压过滤常数\textsuperscript{\cite{ref4,ref5}}的认识,了解恒压过滤模型的影响因素及特性,学会$\frac{d \theta}{d q} - q$一类关系的处理方法,在实际操作过程中对理论模型进行验证。

改进的数字化设计方案需要将实验中部分流程或整个流程进行一定的数学建模,仿真模拟,实现对原有实验的补充和优化,通过信息化设计,整个实验流程得到一定的简化有助于提升学生的学习效果,学习体验。

本章首先通过对实验改进前恒压过滤实验的教学目标以及实验经改进后的教学目标和教学功能进行对比,阐述两者之间的差异性;然后,对数字化设计针对的原型实验原理和内容进行详细的介绍;最后,对数字化设计内容的科学原型、原理、数学建模过程、数据库和算法进行一定的介绍。

\subsection{教学目标分析对比}

% 写明体现实验本身的教学目标及通过数字化设计所能够实现的教学目标和教学功能。

在本节内容中,将实验优化改进前后的教学目标做对比(见表\ref{tab:vs1}),阐述数字化设计如何在原有实验基础上提升学生的知识掌握、技能应用和数据分析能力,使其更好地理解实验内容并应对实际挑战。

\begin{table}[H]
    \centering
    \caption{实验优化前后教学目标对比}  
    \settableinnerfont
    \label{tab:vs1}
    \begin{tabular}{lll}
        \toprule
        \textbf{教学目标} & \textbf{实验优化前} & \textbf{实验优化后} \\
        \midrule
        设备认知与操作流程 & 熟练掌握设备结构、操作流程与注意事项。 & 在前者基础上,学会使用外机设备简化实验流程 \\
        过滤动力学模型构建 & 探究压差影响与速率关系、滤饼压缩性的模型。 & 在前者基础上,探索其他模型或算法。 \\
        模型求解与模型验证 & 手动画图,手动求解 & 自动化处理数据得出结果,生成实验报告 \\
        现代自动化技术应用 & 无过多涉及 & 图像识别技术,数值模拟与实验仿真。 \\
        \bottomrule
    \end{tabular}
\end{table}

\subsubsection{实验优化前拟达到的教学目标}

(1)系统认知板框压滤装置的结构组成与运行机理,熟练掌握该设备的标准化操作流程及注意事项。

(2)建立恒压工况下过滤参数(包括过滤系数$K$、单位面积虚拟滤液量$q_e$及等效过滤时间$\theta_e$)的定量表征方法,深入解析各参数间的物理关联及其调控机制。

(3)探究操作压差对过滤动力学行为的影响规律,构建滤饼可压缩性指标$s$与物料本征常数$k$的耦合计算模型。

(4)学习基于差分法的工艺参数分析方法,通过实验数据建立过滤过程微分项$\frac{d\theta}{dq}$与滤液通量$q$的数学关联。

(5)定量表征滤饼洗涤阶段的传质特性$q$,实验论证终止过滤速率与洗涤速率的动力学关系。

\subsubsection{实验优化后拟达到的教学目标}

(1)系统掌握板框压滤装置的运行机理与构造特征,建立设备操作参数与工艺性能的关联认知,熟练掌握该设备的标准化操作流程及注意事项。

(2)系统解析恒压过滤参数$K$、$q_e$、$\theta_e$的测定方法,通过数值模拟,仿真模拟深化参数物理内涵及其调控机制的认知。

(3)自主构建过滤压差与速率之间的量化关系模型,完善滤饼压缩指数$s$ 与物料本构常数$k$的计算模型。

(4)应用数值微分技术建立$\frac{d\theta}{dq} - q$微分方程模型,基于机器学习算法实现实验曲线的动态拟合与特征参量提取。

(5)定量分析洗涤阶段的传质特性$q$,通过可视化技术完成过滤终了速率与洗涤速率的动力学验证。

(6)了解图像识别基本原理,利用图像识别技术、Python编程语言,实现科研数据或工业大数据可视化,提升在化工单元操作中的系统整合能力与创新应用能力。

通过数字化设计,优化后的教学目标在保持实验原有教学要求的基础上,引入图像识别技术监测滤液水位,通过USB协议将数据传输到中控台上,并通过编程语言流程化处理数据集并生成分析报告。

\subsection{传统恒压过滤实验的原型和原理}

\subsubsection{传统恒压过滤实验的原型}

在未经数字化设计之前,恒压过滤实验的传统流程依赖于人工操作,其步骤如图\ref{fig:exp_de_fc_ori}所示。此流程图描绘了一个简化的实验过程,从实验开始到结束共包含五个主要步骤。实验始于“开始”阶段,学生需手动准备板框压滤装置并配置悬浮液。随后进入“人工观察/高度”步骤,学生通过目视方式监测滤液或滤饼的高度变化,这一过程要求高度专注以确保数据的准确性。接下来是“人工记录数据”阶段,学生手动记录时间(\(\theta\))、滤液体积(\(q\)或\(V\))以及压力差(\(\Delta p\))等关键参数,这些数据是后续分析的基础。在“人工求解”步骤中,学生根据记录的数据,手动绘制\(\frac{\Delta \theta}{\Delta q}\)与\(\bar{q}\)的关系曲线,并通过直线的斜率和截距计算过滤常数\(K\)和等效滤液体积\(q_e\),如式(\ref{eq:constant_p_filtering_constant3})所述。最后,流程以“结束”阶段收尾,实验结果被整理并记录为实验报告。这种传统方法虽然有助于学生理解设备操作和过滤原理,但操作繁琐且易受人为误差影响,尤其在数据记录和计算环节。

\begin{figure}[H]
    \centering
    \includegraphics[width=\textwidth]{figures/exp_de_fc/exp_de_fc_ori.png}
    \caption{恒压过滤实验原型处理流程图}
    \label{fig:exp_de_fc_ori}
\end{figure}

\subsubsection{传统恒压过滤实验的原理}

% 写明数字化设计所针对的原型实验的原理和内容。

恒压过滤过程的数学描述建立在过滤基本方程的基础理论体系之上。而恒压过滤方程作为其具体应用形式,为实验研究提供了关键数学模型支撑,后续过滤常数的测定便是从此出发的。

为了便于理解和使用本文中出现的各种符号及其物理意义,特提供以下符号说明表(见表 \ref{tab:symbols})。该表列出了在恒压过滤实验理论和数据处理中直接涉及的主要符号、其说明及单位。

{
\centering
\settableinnerfont

\begin{longtable}[htbp]{p{3cm}p{8cm}p{3cm}}
    \caption{符号说明}
    \label{tab:symbols} \\

\toprule
    符号 & 说明 & 单位 \\
    \midrule
    \endfirsthead

\multicolumn{3}{c}{{\bfseries (续上表)}} \\
    \toprule
    符号 & 说明 & 单位 \\
    \midrule
    \endhead

\bottomrule
    \endfoot

\bottomrule
    \endlastfoot

\( V \) & 滤液体积 & \(\mathrm{m^3}\) \\
\( A \) & 过滤面积 & \(\mathrm{m^2}\) \\
\( \theta \) & 过滤时间 & \(\mathrm{s}\) \\
\( \Delta p \) & 压力差 & \(\mathrm{Pa}\) \\
\( \mu \) & 滤液黏度 & \(\mathrm{Pa \cdot s}\) \\
\( r \) & 滤饼比阻 & \(\mathrm{m^{-1}}\) \\
\( v \) & 每获得 1 m³ 滤液所形成的滤饼体积 & \(\mathrm{m^3}\) \\
\( V_e \) & 等效滤液体积 & \(\mathrm{m^3}\) \\
\( q \) & 单位过滤面积上的滤液体积 & \(\mathrm{m}\) \\
\( q_e \) & 单位过滤面积上的等效滤液体积 & \(\mathrm{m}\) \\
\( K \) & 恒压过滤常数 & \(\mathrm{m^2 / s}\) \\
\( k \) & 过滤物料特性常数 & \(\mathrm{m^4 / (N \cdot s)}\) \\
\( s \) & 压缩性指数 & 无 \\
\( r' \) & 比阻常数 & \(\mathrm{m^{-1}}\) \\
\( \theta_e \) & 等效过滤时间 & \(\mathrm{s}\) \\

\end{longtable}
}

{\noindent \wuhao \rmfamily \fontspec{Arial} \textbf{2.2.2.1 \quad 恒压过滤常数 $K$、$q_e$ 和 $\theta_e$ 的求取}}
\label{se:yuanli}

对恒压过滤,有

\begin{equation}
    (q + q_e)^2 = K (\theta + \theta_e)
    \label{eq:constant_p_filtering_constant1}
\end{equation}
将式(\ref{eq:constant_p_filtering_constant1})进行微分可得
\begin{equation}
    \frac{d \theta}{d q} = \frac{2}{K} q + \frac{2}{K} q_e
    \label{eq:constant_p_filtering_constant2}
\end{equation}

这是一个直线方程式,由于实验过程中不可能测量到无穷小时间段内的滤液体积的变化,只能测量有限时间段内的滤液体积。当各数据点的时间间隔不大时,$\frac{d \theta}{d q}$ 可用增量之比 $\frac{\Delta \theta}{\Delta q}$ 来代替,于普通坐标上标绘 $\frac{d \theta}{d q} - q$ 的关系曲线:

\begin{equation}
    \frac{\Delta \theta}{\Delta q} = \frac{2}{K} \bar{q} + \frac{2}{K} q_e
    \label{eq:constant_p_filtering_constant3}
\end{equation}

在直角坐标系中,以 $\frac{\Delta \theta}{\Delta q}$ 为纵坐标,相对应的 $\bar{q}$ 为横坐标绘图,可得一直线,直线的斜率为 $\frac{2}{K}$,截距为 $\frac{2}{K} q_e$,从而求出 $K$、$q_e$。至于 $\theta_e$,可由式(\ref{eq:constant_p_filtering_constant4})式求出:

\begin{equation}
    q_e^2 = K \theta_e
    \label{eq:constant_p_filtering_constant4}
\end{equation}

{\noindent \wuhao \rmfamily \fontspec{Arial} \textbf{2.2.2.2 \quad 滤饼的压缩性指数 $s$ 和 物料特性常数 $k$ 的求取}}

改变过滤压力,可得到不同操作压力下的过滤常数 $K$ 值,根据过滤常数的定义式:

\begin{equation}
    K = 2k \Delta p^{1-s}
    \label{eq:constant_p_filtering_constant5}
\end{equation}
两边取对数得
\begin{equation}
    \lg K = (1-s)\lg \Delta p + \lg (2k)
    \label{eq:constant_p_filtering_constant6}
\end{equation}

在不同压力过滤时,由于 $k = \frac{1}{\mu r' v} = \text{常数}$,故 $K$ 与 $\Delta p$ 的关系在对数坐标上标绘时应是一条直线,直线的斜率为 $1-s$,由此可得滤饼的压缩性指数 $s$,然后代入式(\ref{eq:constant_p_filtering_constant6})求物料特性常数 $k$。

\subsection{改进的数字化设计内容的原型和原理}

% 写明数字化设计内容的科学原型和原理、数学物理建模过程、数据库及算法等。

改进的数字化设计针对的具体内容是恒压过滤实验数据采集和处理。在改进的数字化设计过程中,利用opencv识别液位高度,并通过USB协议传输液位高度到控制台,动态获取实验数据,然后依据恒压过滤基本模型,根据数据形状进行一定的数学建模、算法设计,通过数值仿真、可视化等手段将实验结果呈现在平台上。

\subsubsection{改进的数字化设计内容的传统实验原型}

改进后经数字化设计的实验流程引入了自动化和信息化技术,大幅优化了数据采集与处理效率,其步骤如图\ref{fig:exp_de_fc}所示。此流程图展示了一个更为复杂的实验过程,从“开始”到“结束”包含多个并行和顺序步骤,并以“ChemLabX”系统为核心处理单元。实验初始阶段为“原始数据采集”,利用图像识别技术(如OpenCV)自动监测滤液高度,取代了人工观察。随后进入“读取数据,数据清洗滤波”步骤,通过算法去除数据中的噪声和异常值。流程后续“计算\(\Delta \theta / \Delta q\)”,直接计算时间变化与滤液体积变化的比率;进入“拟合模型”阶段,通过数学模型(式\ref{eq:constant_p_filtering_constant3})拟合数据,可选择插值法或经验公式以提高精度。进一步分析得出滤饼压缩指数\(s\),利用经验关系\(\lg K = (1 - s) \lg \Delta p + \lg (2k)\)(式\ref{eq:constant_p_filtering_constant6}),最终数据通过USB传输至“ChemLabX”系统进行处理和可视化,生成实验结果并结束流程。

\begin{figure}[H]
    \centering
    \includegraphics[width=\textwidth]{figures/exp_de_fc/exp_de_fc.png}
    \caption{改进的数字化设计恒压过滤实验处理流程图}
    \label{fig:exp_de_fc}
\end{figure}

改进的数字化设计的内容的核心是数据的采集和处理。通过自动化设备获得一系列滤液高度情况下所需过滤的时间,求算出在这段变化时间内液体压力的变化\(\Delta p\)以及时间的变化\(\theta\),然后可以通过式(\ref{eq:constant_p_filtering_constant3})、式(\ref{eq:constant_p_filtering_constant4})、式(\ref{eq:constant_p_filtering_constant5})、式(\ref{eq:constant_p_filtering_constant6})求得恒压过滤常数,滤饼压缩系数以及物料特性常数。

\subsubsection{改进的数字化设计内容的原理}

{\noindent \wuhao \rmfamily \fontspec{Arial} \textbf{2.3.2.1 \quad 数据采集}}

%%%  原始图像处理
% 1)图像采集
% 2) 二值化

%%%  水槽定位
% 3) 增强对比度
% 4) 形态学处理
% 5) 边缘检测实现水槽定位

%%%  获取结果
% 6) 进一步边缘检测实现液位高度
% 7) 保存数据

在数字化设计中,原始数据采集是实验能够自动化的必要条件。过滤过程中,首先通过使用外接摄像头,实时获取滤缸(透明塑料材质且带有刻度)内液位高度的变化的图片,并将监控的图片二值化;然后,采取增强对比度,形态学分析,边缘检测等算法定标水槽位置;再者,通过传统边缘检测或可选的卷积神经网络(CNN)回归模型确定水位数据,并将测得的液位数据记录于文件中;最后,将原始数据的记录文件传输至数据处理层进一步处理获取数据处理结果和可视化结果。

\begin{figure}[H]
    \centering
    \includegraphics[width=1.0\textwidth]{figures/camera/camera.png}
    \caption{摄像头实时监测示意图}
    \label{fig:camera}
\end{figure}

1.\textbf{图像采集}: 使用外接摄像头实时监测滤缸内液位高度的变化情况。摄像头通过 USB 接口连接到计算机,实时捕捉滤缸内液位高度的变化。实验数据采集过程示意图如图\ref{fig:camera} 所示。采集的实验装备图像如图\ref{fig:equipment}所示。

\begin{figure}[H]
    \centering
    \includegraphics[width=0.90\textwidth]{figures/equipment/equipment.png}
    \caption{实验装备采集图}
    \label{fig:equipment}
\end{figure}

2.\textbf{图像二值化}:使用图像处理库(如 OpenCV)对采集到的图像进行二值化处理。通过设置合适的阈值,将图像转换为黑白二值图像,以便后续的水槽定位、边缘检测和液位高度识别\textsuperscript{\cite{ref13}}。

3.\textbf{增强对比度}:对二值化后的图像进行增强对比度处理\textsuperscript{\cite{ref14}},以提高水槽的可见性。可以使用直方图均衡化等方法来增强图像的对比度。

4.\textbf{形态学处理}:使用形态学处理方法(如开运算、闭运算)对二值化后的图像进行处理,以去除噪声和小的干扰物体。通过设置合适的结构元素和迭代次数,确保水槽的清晰可见。

5.\textbf{边缘检测-1}:使用边缘检测算法(如 Canny 算法)对二值化后的图像进行处理,提取水槽的边缘信息。通过设置合适的阈值,确保水槽边缘的准确识别。

步骤2-5的处理结果如图\ref{fig:combined_result_2.png} 所示:

\begin{figure}[H]
    \centering
    \includegraphics[width=0.90\textwidth]{figures/bina/combined_result_2.png}
    \caption{步骤2-5的处理结果}
    \label{fig:combined_result_2.png}
\end{figure}

6.\textbf{液位高度识别}:对上述处理结果中的图进行液位高度识别。传统方法通过第二次边缘检测(边缘检测-2)获取液位数据,处理后的图像如图\ref{fig:combined_result_1.png}所示。为进一步提高准确性和鲁棒性,提出使用卷积神经网络(CNN)回归模型作为可选方法。CNN模型接收裁剪后的感兴趣区域(ROI)图像,预测液位高度(像素单位),随后通过定标系数转换为实际高度。CNN模型需在标注数据集上训练,学习图像与液位高度的映射关系。

\begin{figure}[H]
    \centering
    \includegraphics[width=0.90\textwidth]{figures/bina/combined_result_1.png}
    \caption{边缘检测获取液位数据}
    \label{fig:combined_result_1.png}
\end{figure}

7.\textbf{数据保存}:将识别到的一系列液位高度数据(来自传统边缘检测或CNN预测)以及对应的过滤时间数据保存为csv文件,便于后续的数据处理,数据记录格式如图\ref{fig:csv_data}所示。

\begin{figure}[H]
    \centering
    \includegraphics[width=0.90\textwidth]{figures/csv_data/csv_data.png}
    \caption{csv数据记录格式}
    \label{fig:csv_data}
\end{figure}

{\noindent \wuhao \rmfamily \fontspec{Arial} \textbf{2.3.2.2 \quad 数据处理}}

% 数据清洗,数据处理,数据拟合,异常值检测,数据可视化

在上述获得实验数据的基础上,使用Python进行编程,首先对得到的原始数据进行数据清洗和异常值检测;然后,按照恒压过滤实验原型的理论依据,通过式(\ref{eq:constant_p_filtering_constant3})、式(\ref{eq:constant_p_filtering_constant4})、式(\ref{eq:constant_p_filtering_constant5})、式(\ref{eq:constant_p_filtering_constant6})进行编程式计算,完成数据处理;最后,对得到的数据处理结果进行可视化工作。

1.\textbf{数据清洗}:对原始数据进行清洗,去除无效值(如NaN),确保数据质量。

2.\textbf{异常值检测}:通过Huber回归的残差和Z-score方法检测异常值。计算每个数据点的残差,并基于Z-score阈值识别异常值。若检测到异常值,则保存未去除异常值的数据一份,同时保存去除异常值后的数据一份,为后续可视化对比准备。

3.\textbf{数据拟合}:使用Huber回归模型对处理后的数据进行鲁棒拟合,得到恒压过滤常数 \(K\)、虚拟滤液量 \(q_e\) 和等效过滤时间 \(\theta_e\)。Huber回归通过结合L1和L2损失的优势,减少异常值对拟合结果的影响。

4.\textbf{数据处理}:将清洗后的数据进行处理,计算液位高度随时间变化的关系。

5.\textbf{数据可视化}:将处理后的数据和拟合结果进行可视化,帮助分析过滤过程的动态变化,并对比去除异常值前后的结果。

\subsubsection{改进的数字化设计的数学物理建模过程}

在数字化设计恒压过滤实验的过程中,首先通过摄像设备实时监测物理过程,并将其抽象化。通过图像二值化、增强对比度、形态学处理和边缘检测等技术,定位水槽位置。在定位完成后,进行第二次边缘检测以测量水槽内滤液高度的尺寸,之后将这些数据保存到文件中,并通过编程好的软件进行处理和计算。

在软件内部的处理过程中,首先进行数据清洗,去除无效值。接下来,通过Huber回归模型进行鲁棒拟合,并基于残差的Z-score方法检测异常值。Huber回归通过最小化Huber损失函数,确保拟合对异常值具有鲁棒性,而残差Z-score方法进一步提升异常值检测的精度。处理后的数据用于计算关键参数,包括过滤常数(\(K\))、滤饼压缩指数(\(s\))和物料特性常数(\(k\))。最后,通过数值仿真与可视化技术,处理后的数据结果在终端上呈现,直观展示过滤过程的动态特性。

\subsubsection{改进的数字化设计的数学物理建模过程}

% 写明数学物理建模过程

在数字化设计恒压过滤实验的过程中,首先需要通过摄像设备实时监测物理过程,并将其抽象化。通过图像二值化、增强对比度以及形态学处理和边缘检测等技术,定位水槽的位置。在定位完成后,进行第二次边缘检测以测量水槽内滤液高度的尺寸,之后将这些数据保存到文件中,并通过编程好的软件进行处理和计算。

在软件内部的处理过程中,首先进行数据清洗,去除无效值。接下来,通过Z-score方法检测异常值,并将处理后的数据拷贝成两份。一份数据用于正常拟合,另一份数据则用于去除异常值后的拟合。过程中,依据恒压过滤的基本原理进行数学建模,计算出过滤常数(K)、滤饼压缩指数(s)和物料特性常数(k)等关键参数。最后,通过数值仿真与可视化技术,处理后的数据结果在终端上呈现。

以下是相关描述的数学建模过程:

{\noindent \wuhao \rmfamily \fontspec{Arial} \textbf{2.3.4.1 \quad 数据采集过程的数学建模}}

在恒压过滤实验中,数据采集过程的数学建模可以通过以下几个步骤来描述。首先,使用摄像设备实时监测水槽中滤液的高度变化,并通过传统图像处理或CNN模型处理图像数据。

1. 图像二值化与对比度增强

图像二值化过程将原始图像 \( I(x, y) \) 转换为黑白图像。设定一个阈值 \( T \),对图像进行二值化处理,得到二值图像 \( I_b(x, y) \):

\begin{equation}
I_b(x, y) = 
\begin{cases} 
1, & \text{if } I(x, y) \geq T \\
0, & \text{if } I(x, y) < T
\end{cases}
\end{equation}

接下来,进行对比度增强,增强后的图像 \( I_e(x, y) \) 可以通过对比度限制自适应直方图均衡化\textsuperscript{\cite{ref15,ref18}}(CL\\AHE)来进行,增强的公式为:

\begin{equation}
I_e(x, y) = \text{CLAHE}(I(x, y), \text{clipLimit}, \text{tileGridSize})
\end{equation}

2. 形态学处理与边缘检测

在图像增强之后,使用形态学运算(如开运算和闭运算)去除噪声\textsuperscript{\cite{ref19}},得到处理后的图像 \( I_p(x, y) \):

\begin{equation}
I_p(x, y) = \text{Morphology}(I_e(x, y))
\end{equation}

接着,应用边缘检测算法\textsuperscript{\cite{ref16,ref17}}(如Canny边缘检测),得到图像中的边缘信息 \( E(x, y) \):

\begin{equation}
E(x, y) = \text{Canny}(I_p(x, y), T_1, T_2)
\end{equation}

其中,\( T_1 \) 和 \( T_2 \) 是边缘检测的两个阈值。

3. 水槽定位与液位高度测量

在边缘检测后,根据检测到的轮廓计算水槽的实际位置 \( (x_s, y_s) \),并测量水槽内滤液的高度 \( h \)。设水槽的实际宽度为 \( W_{\text{real}} \)(单位:cm),计算像素到厘米的转换系数 \( C_{\text{px2cm}} \):

\begin{equation}
C_{\text{px2cm}} = \frac{W_{\text{real}}}{W_{\text{px}}}
\end{equation}

其中,\( W_{\text{px}} \) 是水槽在图像中的宽度(单位:像素)。

传统方法通过水槽的边界,测量滤液的高度 \( h_{\text{px}} \)(单位:像素),并转换为实际高度 \( h_{\text{real}} \)(单位:cm):

\begin{equation}
h_{\text{real}} = h_{\text{px}} \cdot C_{\text{px2cm}}
\end{equation}

为提高测量精度,提出使用卷积神经网络\textsuperscript{\cite{ref20,ref21}}(CNN)回归模型预测液位高度。CNN模型 \( f_{\theta} \) 接收预处理后的ROI图像 \( I_{\text{ROI}} \),输出液位高度 \( h_{\text{px}} \):

\begin{equation}
h_{\text{px}} = f_{\theta}(I_{\text{ROI}})
\end{equation}

其中,\( \theta \) 为模型参数。CNN模型由多层卷积层、池化层和全连接层组成,通过最小化均方误差损失函数训练:

\begin{equation}
\mathcal{L} = \frac{1}{N} \sum_{i=1}^{N} (h_i - \hat{h}_i)^2
\end{equation}

其中,\( N \) 为训练样本数,\( h_i \) 为真实液位,\( \hat{h}_i \) 为预测液位。预测的像素高度 \( h_{\text{px}} \) 转换为实际高度 \( h_{\text{real}} \) 使用上述公式。

4. 数据保存与传输

上述处理步骤后,获取的滤液高度 \( h_{\text{real}} \)(来自传统方法或CNN预测)和相应的时间数据 \( t \) 将保存为数据文件,并传输至计算系统进行后续处理。

{\noindent \wuhao \rmfamily \fontspec{Arial} \textbf{2.3.4.2 \quad 软件内部数据处理过程的数学建模}}

在数字化设计的过程中,处理实验数据的核心部分是通过数学模型对实验数据进行拟合、异常值检测以及模型计算。具体过程如下:

1. 数据加载与预处理

在数据加载后,首先对实验数据进行预处理(其中包含液位高度和时间变化的数据)。通过删除缺失值(NaN)以及将数据转换为数值型,确保数据的质量。将加载的数据存储在一个二维数据矩阵 \( D \),其中第 \( i \) 行是时间步长 \( \theta_i \) 和对应的液位高度 \( q_i \):

\begin{equation}
D = \begin{pmatrix}
\theta_1 & q_1 \\
\theta_2 & q_2 \\
\vdots & \vdots \\
\theta_n & q_n
\end{pmatrix}
\end{equation}

2. 鲁棒线性拟合

在对数据进行清洗后,进行鲁棒线性拟合。为了得到恒压过滤常数 \( K \),基于恒压过滤理论,对实验数据进行处理。通过Huber回归模型\textsuperscript{\cite{ref23}}对 \( \frac{\Delta \theta}{\Delta q} \) 和 \( q \) 进行拟合,得到线性关系:

\begin{equation}
\frac{\Delta \theta}{\Delta q} = \text{slope} \cdot q + \text{intercept}
\end{equation}

Huber回归通过最小化Huber损失函数来估计斜率和截距,对异常值具有鲁棒性。恒压过滤常数 \( K \) 可由斜率计算得出:

\begin{equation}
\hat{K} = \frac{1}{\text{slope}}
\end{equation}

3. 异常值检测与数据清洗

在初拟合后,基于鲁棒回归的残差进行异常值检测。计算每个数据点的残差 \( r_i = y_i - \hat{y}_i \),并使用Z-score方法识别异常值:

\begin{equation}
Z = \frac{|r_i|}{\sigma_r}
\end{equation}

其中,\( \sigma_r \) 是残差的标准差。如果 \( Z \) 大于预设阈值,则认为该数据点是异常值。检测到异常值后,程序会将其从数据中剔除,然后使用标准线性回归重新进行拟合。

4. 数据拟合后的计算

在移除异常值并重新拟合后,程序将再次应用线性回归模型进行拟合,得到新的恒压过滤常数 \( K \)、滤饼压缩指数 \( s \) 和物料特性常数 \( k \):

\begin{equation}
K = \frac{1}{\text{slope}}, \quad s = \text{由实验确定}, \quad k = \text{物料常数}
\end{equation}

5. 数值仿真与可视化

最终处理结果图\textsuperscript{\cite{ref5,ref6}}如图\ref{fig:fit_res_a}、图\ref{fig:fit_res_b}、图\ref{fig:fit_res_c}、图\ref{fig:fit_res_d}所示(部分):

\begin{figure}[H]
    \centering
    \begin{subfigure}[b]{0.45\textwidth}
        \centering
        \includegraphics[width=\textwidth]{figures/fit_res/fit_res_a.png}
        \caption{拟合图结果a}
        \label{fig:fit_res_a}
    \end{subfigure}
    \hfill
    \begin{subfigure}[b]{0.45\textwidth}
        \centering
        \includegraphics[width=\textwidth]{figures/fit_res/fit_res_b.png}
        \caption{拟合图结果b}
        \label{fig:fit_res_b}
    \end{subfigure}
    \vskip\baselineskip
    \begin{subfigure}[b]{0.45\textwidth}
        \centering
        \includegraphics[width=\textwidth]{figures/fit_res/fit_res_c.png}
        \caption{拟合图结果c}
        \label{fig:fit_res_c}
    \end{subfigure}
    \hfill
    \begin{subfigure}[b]{0.45\textwidth}
        \centering
        \includegraphics[width=\textwidth]{figures/fit_res/fit_res_d.png}
        \caption{拟合图结果d}
        \label{fig:fit_res_d}
    \end{subfigure}
    \caption{拟合结果(部分)}
    \label{fig:fit_res}
\end{figure}

所有图像的整合结果如图\ref{fig:fit_res_intergrated}所示。

\begin{figure}[H]
    \centering
    \includegraphics[width=0.95\textwidth, height=0.95\textheight, keepaspectratio]{figures/fit_res/fit_res.png}
    \caption{拟合图整合图}
    \label{fig:fit_res_intergrated}
\end{figure}

\pagebreak{}

6. 最终结果与输出

所有处理后的数据将输出为CSV文件,包括滤液体积、过滤时间和恒压过滤常数等参数。数据处理结果会被保存并展示在图形界面中,帮助学生理解恒压过滤过程中的各项物理量之间的关系。

\subsubsection{数字化设计的数据库及算法设计}

{\noindent \wuhao \rmfamily \fontspec{Arial} \textbf{2.3.5.1 \quad 数据采集算法设计}}

\begin{breakablealgorithm}
    \caption{数据采集数学建模}
    \begin{algorithmic}[1]
        \Require 图像数据 $I(x, y)$
        \Ensure 处理后的图像数据 $I_b(x, y)$, $I_e(x, y)$, $I_p(x, y)$,边缘检测结果 $E(x, y)$,水槽位置 $(x_s, y_s)$,液体高度 $h_{\text{real}}$
        
        % 图像二值化和对比度增强
        \Function{Image Binarization and Contrast Enhancement}{$I(x, y)$}
            \State 设置阈值 $T$
            \State 生成二值化图像 $I_b(x, y)$:
            \[
            I_b(x, y) = 
            \begin{cases} 
            1, & \text{如果 } I(x, y) \geq T \\
            0, & \text{如果 } I(x, y) < T
            \end{cases}
            \]
            \State 对图像进行增强,得到 $I_e(x, y)$:
            \[
            I_e(x, y) = \text{CLAHE}(I(x, y), \text{clipLimit}, \text{tileGridSize})
            \]
        \EndFunction

        % 图像形态学处理和边缘检测
        \Function{Morphological Processing and Edge Detection}{$I_e(x, y)$}
            \State 进行形态学操作,得到处理后的图像 $I_p(x, y)$:
            \[
            I_p(x, y) = \text{Morphology}(I_e(x, y))
            \]
            \State 应用 Canny 边缘检测,得到边缘检测结果 $E(x, y)$:
            \[
            E(x, y) = \text{Canny}(I_p(x, y), T_1, T_2)
            \]
        \EndFunction

        % 水槽位置和液体高度测量
        \Function{Tank Position and Liquid Height Measurement}{$E(x, y)$}
            \State 计算水槽位置 $(x_s, y_s)$
            \State 设置实际水槽宽度 $W_{\text{real}}$
            \State 计算像素到厘米的转换系数 $C_{\text{px2cm}}$:
            \[
            C_{\text{px2cm}} = \frac{W_{\text{real}}}{W_{\text{px}}}
            \]
            \State 测量液体高度 $h_{\text{px}}$(单位:像素),并转换为实际高度 $h_{\text{real}}$:
            \[
            h_{\text{real}} = h_{\text{px}} \cdot C_{\text{px2cm}}
            \]
        \EndFunction
        
        \Function{Data Saving and Transmission}{$h_{\text{real}}, t$}
            \State 保存数据并传输到计算系统进行进一步处理
        \EndFunction
    \end{algorithmic}
\end{breakablealgorithm}

改进的恒压过滤实验图像数据采集及数据处理算法流程图见图\ref{fig:algo1},其核心目的是处理从实验中获取的图像数据,并通过一系列图像处理技术提取液体容器的液位信息。

\begin{figure}[H]
    \centering
    \includegraphics[width=0.90\textwidth, height=0.95\textheight]{figures/algo/algo1.png}
    \caption{数据采集算法流程图}
    \label{fig:algo1}
\end{figure}

从图\ref{fig:algo1}中可看出,算法首先通过加载实验图像并选择感兴趣区域(ROI)开始处理。接着,通过CLAHE增强对比度、高斯模糊平滑图像、Canny边缘检测提取特征,并应用形态学闭运算去除噪声。随后,算法寻找并选择最佳轮廓以定位水槽,计算像素到厘米的定标系数。液位高度检测支持两种方法:传统方法通过第二次Canny边缘检测和行和最大值计算高度;可选的卷积神经网络(CNN)回归模型通过预训练模型预测高度,提高鲁棒性。最终,高度转换为实际单位,可视化并保存为CSV文件,传输至后续处理模块。

{\noindent \wuhao \rmfamily \fontspec{Arial} \textbf{2.3.5.2 \quad 软件内部数据处理算法设计}}

\begin{breakablealgorithm}
    \caption{软件内部数据处理数学建模}
    \begin{algorithmic}[1]
        \Require 实验数据矩阵 $D$
        \Ensure 过滤后的数据和模型参数 $K$, $s$, $k$
        
        \Function{Data Loading and Preprocessing}{$D$}
            \State 加载CSV数据,清洗数据,去除缺失值
            \State 将数据存储在二维矩阵 $D$ 中,其中第 $i$ 行表示时间步 $\theta_i$ 和液体高度 $q_i$:
            \[
            D = \begin{pmatrix}
            \theta_1 & q_1 \\
            \theta_2 & q_2 \\
            \vdots & \vdots \\
            \theta_n & q_n
            \end{pmatrix}
            \]
        \EndFunction
        
        \Function{Robust Linear Fitting}{$D$}
            \State 计算液体体积 $V$ 与时间之间的关系
            \State 基于恒压过滤方程,执行以下计算:
            \[
            \frac{\Delta \theta}{\Delta q} = \frac{K A^2}{2(V + V_e)}
            \]
            \State 使用Huber回归模型对数据进行拟合,求解恒压过滤常数 $K$:
            \[
            \frac{\Delta \theta}{\Delta q} = \text{slope} \cdot q + \text{intercept}
            \]
            \State 通过Huber损失函数估计斜率和截距,得到初步 $K$:
            \[
            \hat{K} = \frac{1}{\text{slope}}
            \]
        \EndFunction
        
        \Function{Outlier Detection and Data Cleaning}{$D$}
            \State 计算残差 $r_i = y_i - \hat{y}_i$,并使用Z分数方法:
            \[
            Z = \frac{|r_i|}{\sigma_r}
            \]
            \State 若 $Z$ 大于阈值(默认2),则移除该数据点
        \EndFunction
        
        \Function{Post-Fitting Calculations}{$D$}
            \State 移除异常值后,使用标准线性回归拟合,得到最终 $K$、$s$、$k$:
            \[
            K = \frac{1}{\text{slope}}, \quad s = \text{由实验确定}, \quad k = \text{物料常数}
            \]
        \EndFunction
        
        \Function{Numerical Simulation and Visualization}{$D$}
            \State 使用数学模型进行数值仿真,预测液体体积和高度
            \State 可视化拟合结果与实验数据,比较异常值影响
        \EndFunction
        
        \Function{Final Results and Output}{$K, s, k$}
            \State 输出液体体积、过滤时间和过滤常数
        \EndFunction
    \end{algorithmic}
\end{breakablealgorithm}

改进的过滤实验数据处理算法流程图见图\ref{fig:algo2},该算法通过数学建模提取液位数据,拟合出恒压过滤常数 $K$、滤饼压缩指数 $s$ 和物料特性常数 $k$。

\begin{figure}[H]
    \centering
    \includegraphics[width=0.83\textwidth, height=0.83\textheight,keepaspectratio]{figures/algo/algo2.png}
    \caption{数据处理算法流程图}
    \label{fig:algo2}
\end{figure}

从图\ref{fig:algo2}中可看出,算法首先加载实验数据并进行预处理,去除缺失值。接着,初始化组索引,逐组处理数据。每组数据通过提取、变量计算、Huber回归鲁棒拟合、基于残差的Z分数异常值检测、线性回归二次拟合等步骤,计算过滤常数 $K$。异常值检测确保拟合精度,二次拟合生成最终参数。所有组处理完成后,输出结果并保存为CSV文件,完成过滤过程分析。