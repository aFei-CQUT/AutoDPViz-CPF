% ==================== 正文字号 ====================

\wuhao                      % 字号:五号
\rmfamily                   % 中文:宋体
\fontspec{Times New Roman}  % 英文:Times New Roman

% ====================== 引言 ======================

\section{引言}

化工原理恒压过滤实验是化工专业核心基础课程的重要组成部分,不仅是教育部指定的教学内容,更是学生理解化工单元操作、培养实际动手能力和理论应用能力的关键环节。通过指导学生完成实验操作并撰写实验报告,教师能够有效提升学生的专业技能,加深其对理论知识的理解与吸收。然而,传统恒压过滤实验教学面临显著挑战:数据采集耗时长(通常持续4-6小时)、数据处理流程繁琐,不仅占用了学生大量的时间和精力,也在一定程度上限制了学生的学习体验。

随着数字信息化技术的迅猛发展,自动化与智能化技术正深刻改变教育领域\textsuperscript{\cite{ref1}}。将先进数字化技术融入传统实验教学,对现有实验装置和流程进行优化,使学生能够利用高效工具完成实验操作,已成为当前教育数字化转型的重要方向。化工实验教学的数字化不仅能够提升效率,还能为学生提供数据驱动的学习体验,推动教学方式从标准化向个性化和智能化转变。

针对传统恒压过滤实验\textsuperscript{\cite{ref2,ref3}}的上述问题,本研究提出了一种高度自动化的实验处理方案。通过构建软硬件协同的智能实验平台,实现了实验数据采集、处理及图表生成的全流程自动化。具体而言,本研究基于OpenCV图像识别技术和Python编程语言,开发了一款名为ChemLabX的化工实验数据采集与处理软件。该软件利用图像识别技术实时监测滤液水位,自动记录并处理实验数据,显著简化了人工操作环节,将学生在数据采集与处理上花费的时间缩短至传统方法的1/3以内,从而大幅提升了实验效率和学生参与度。

ChemLabX软件(见下页图\ref{fig:software_calculated})的创新性不仅体现在其高效解决恒压过滤实验数据处理难题,还在于其卓越的通用性和可扩展性。该软件设计了一个模块化框架,不仅适用于恒压过滤实验,还可无缝应用于化工原理课程中的其他典型实验,如蒸馏、萃取等,目前已集成了7个实验,为实验教学的全面数字化提供了技术支持。此外,本研究提出的自动化处理方案形成了一种可复现的结构和方法论,为后续实验流程的智能化改造奠定了基础。这种数据驱动的实验教学模式,不仅优化了当前教学体验,还为相关领域的数字化应用提供了有益探索。

\begin{figure}[H]
\centering
\includegraphics[width=\textwidth,keepaspectratio]{figures/software_gui/software_calculated.png}
\caption{ChemLabX 软件主界面及运行后结果}
\label{fig:software_calculated}
\end{figure}

图\ref{fig:software_calculated}展示了ChemLabX软件的主界面及运行结果。

在未进行数据导入或者数据采集的情况下,界面左侧6个按键下方的第一个显示栏(原始数据预览区)、第二个显示栏(实验数据处理结果预览区)以及右侧绘图预览区均为空白。此时为软件的初始状态。

后续有两种方式供学生完成实验数据处理,下面是对两种数据处理方式的详细叙述:

一种是将记录的实验数据通过“导入数据”按钮导入,导入的数据会自动显示到原始数据预览区供使用者校对。接下来,使用者可以依次点击“处理数据”、“绘制图形”按钮,软件会自动完成原始数据的处理以及实验数据的可视化处理,并将结果输出到实验数据处理结果预览区或可视化结果预览区。

另一种是连接外用设备,同时也是本设计方案想要达到的效果,即外接图像采集设备通过USB连接到计算机,依次点击“打开串口”、“开始采集”、“停止采集”,软件会按照预设定的实验采集配置(在CONFIG文件中配置)完成图像的采集以及图像数据的识别并将结果显示到原始数据预览区。接下来,使用者可以依次点击“处理数据”、“绘制图形”按钮,完成后续操作。

ChemLabX软件采用模块化设计,主要包括应用管理模块(App.py),负责管理各类实验的屏幕界面;数据处理模块,包含实验处理器、计算器和绘图器,负责实验数据的计算和可视化;以及工具模块,提供通用功能支持。每个实验的屏幕界面基于通用的基础屏幕类(BaseScreen)构建,而数据处理则通过特定的实验处理器调用相应的计算器和绘图器实现。具体的架构信息如图\ref{fig:software_class}所示。

\begin{figure}[H]
\centering
\includegraphics[width=0.90\textwidth,keepaspectratio]{figures/app_class/app_class.png}
\caption{ChemLabX 软件模块}
\label{fig:software_class}
\end{figure}

需要特别指出的是,本优化方案并未减少实验本身所需的时间,而是通过自动化技术显著减少了学生在数据采集和处理环节所需投入的时间和精力。在传统实验中,学生需要全程参与数据采集和处理,耗费大量时间。而ChemLabX软件的引入,使这些重复性、耗时的操作得以自动化完成,从而解放了学生的时间,使他们能够将精力更多地投入到实验现象的观察、理论知识的理解以及实验报告的撰写等高阶学习活动中。这种优化方式不仅提升了实验教学的效率,还增强了学生的学习体验和自主学习能力,为教学过程注入了新的活力。