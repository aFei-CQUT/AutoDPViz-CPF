% ====================== 参考文献 ======================

\pagebreak{}

{\liuhao

\begin{thebibliography}{99}   % 适用于 1-99 条参考文献

    \bibitem{ref1} 迟松江, 路战红, 张秀红. 自动化过滤常数测定装置及应用[J]. 沈阳工业大学学报, 2001, 28(03): 255-257.	(中文期刊)

    \bibitem{ref2} 张春芝, 贾金霖, 潘志强, 等. 板框式过滤机过滤常数测定研究[J]. 北京工商大学学报(自然科学版), 2010, 28(03): 68-71.	(中文期刊)

    \bibitem{ref3} 赵帅, 郑婷婷, 李傲雪, 等. 恒压过滤常数测定实验的影响因素探究[J]. 长春师范大学学报, 2022, 41(08): 99-106.	(中文期刊)

    \bibitem{ref4} 葛艳蕊, 顾丽敏, 曹亚鹏. 恒压过滤实验中测定值偏离理论值的原因探讨[J]. 实验室科学, 2024, 27(01): 54-57.	(中文期刊)

    \bibitem{ref5} 杨鹰, 李海普, 满瑞林. 恒压过滤常数的实验数据处理[J]. 广州化工, 2018, 46(18): 113-115.	(中文期刊)

    \bibitem{ref6} 张习博. 基于Origin软件的化工原理实验曲线拟合方法与应用[J]. 粘接, 2019, 40(09): 122-126.	(中文期刊)

    \bibitem{ref7} 宋莎, 黎亚明, 王艳力, 等. 自制恒压过滤参数测定实验多功能设备的实践[J]. 实验技术与管理, 2015(06): 94-96+142.	(中文期刊)

    \bibitem{ref8} 吴燕翔, 王碧玉. 恒压过滤介质阻力分析[J]. 化工学报, 2004, 12(01): 33-36.	(中文期刊)

    \bibitem{ref9} 苏旭霞, 刘素楠, 管立新. 开展虚拟仿真实验,促进实验教学改革[J]. 科技广场, 2008(03): 252-254.	(中文期刊)

    \bibitem{ref10} 潘正官, 谢佑国. 自行研制实验设备创建高水平示范性实验室[J]. 实验室研究与探索, 2006, 25(09): 1152-1153.	(中文期刊)

    \bibitem{ref11} 胡芳, 赵亮方. 高校自制实验教学设备的体会[J]. 中国现代教育装备, 2011(15): 42-43.	(中文期刊)

    \bibitem{ref12} 廖庆敏, 秦钢年, 蒙艳玫. 把自制设备作为实验教学示范中心的特色[J].	(中文期刊)

    \bibitem{ref13} Nguyen, T. P. H.; Cai, Z.; Nguyen, K.; Keth, S.; Shen, N.; Park, M. Pre-processing Image using Brightening, CLAHE and RETINEX. \textit{arXiv} 2020, arXiv:2003.10822. https://doi.org/10.48550/arXiv.2003.10822	(英文期刊)

    \bibitem{ref14} Kryjak, T.; Blachut, K.; Szolc, H.; Wasala, M. Real-Time CLAHE Algorithm Implementation in SoC FPGA Device for 4K UHD Video Stream. \textit{Electronics} 2022, 11(14), 2248. https://doi.org/10.3390/electronics11142248	(英文期刊)

    \bibitem{ref15} Mishra, A. Contrast Limited Adaptive Histogram Equalization (CLAHE) Approach for Enhancement of the Microstructures of Friction Stir Welded Joints. \textit{arXiv} 2021, arXiv:2109.00886. https://doi.org/10.48550/arXiv.2109.00886	(英文期刊)

    \bibitem{ref16} Akbari Sekehravani, E.; Babulak, E.; Masoodi, M. Implementing canny edge detection algorithm for noisy image. \textit{Bulletin of Electrical Engineering and Informatics}, 2020, 9(4), 1404–1410. https://doi.org/10.11591/eei.v9i4.1837	(英文期刊)

    \bibitem{ref17} Agrawal, H.; Desai, K. CANNY EDGE DETECTION: A COMPREHENSIVE REVIEW. \textit{International Journal of Technical Research \& Science}, 2024, 9(Spl), 27–35. https://doi.org/10.30780/specialissue-ISET-2024/023	(英文期刊)

    \bibitem{ref18} Jamil, N.; Sembok, T.; Bakar, Z. Noise removal and enhancement of binary images using morphological operations. In \textit{Int Symp Inf Technol 2008 ITSim}, 2008, 3, 1-6. https://doi.org/10.1109/ITSIM.2008.4631954	(英文会议)

    \bibitem{ref19} Horebeek, J.; Tapia-Rodriguez, E. The approximation of a morphological opening and closing in the presence of noise. \textit{Signal Processing}, 2001, 81, 1991-1995. https://doi.org/10.1016/S0165-1684(01)00060-3	(英文期刊)

    \bibitem{ref20} Olbrich, M.; Riazy, L.; Kretz, T.; Leonard, T.; van Putten, D. S.; Bär, M.; Oberleithner, K.; Schmelter, S. Deep learning based liquid level extraction from video observations of gas–liquid flows. \textit{Int. J. Multiphase Flow}, 2022, 157, 104247. https://doi.org/10.1016/j.ijmultiphaseflow.2022.104247	(英文期刊)

    \bibitem{ref21} Dou, G.; Chen, R.; Han, C.; Liu, Z.; Liu, J. Research on Water-Level Recognition Method Based on Image Processing and Convolutional Neural Networks. \textit{Water}, 2022, 14(12), 1890. https://doi.org/10.3390/w14121890	(英文期刊)

    \bibitem{ref22} Nicholaus, I. T.; Lee, J.-S.; Kang, D.-K. One-Class Convolutional Neural Networks for Water-Level Anomaly Detection. \textit{Sensors}, 2022, 22(22), 8764. https://doi.org/10.3390/s22228764	(英文期刊)

    \bibitem{ref23} Sun, Q.; Zhou, W. X.; Fan, J. Adaptive Huber Regression. \textit{J. Am. Stat. Assoc.}, 2020, 115(529), 254-265. https://doi.org/10.1080/01621459.2018.1543124	(英文期刊)

\end{thebibliography}

}