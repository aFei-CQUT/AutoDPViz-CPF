% ====================== 中文摘要 ======================

% 中文摘要
\noindent\textbf{摘要:}\abstractcn{针对传统化工原理恒压过滤实验中人工数据记录耗时长、数据处理流程繁琐的问题,本设计方案以数字化技术为核心,构建了一套软硬件协同的智能实验平台。基于数字化图像识别技术(OpenCV),实现了滤液水槽的自动定标、液位实时监控与数据存储;通过Python开发了数字化实验数据处理系统(ChemLabX),结合恒压过滤实验需求动态筛选数据子集,利用差分法计算$\Delta\theta/\Delta q$序列,并引入Z-score异常值检测机制实时剔除离群点;进一步依托Scikit-learn机器学习框架建立线性回归模型,对过滤方程进行数字化重拟合以提升模型精度。最终,系统将处理结果与可视化图表集成至用户交互界面。该方案通过实验流程的数字化重构,显著简化了传统人工操作环节,具有 高精度、高效率、可复现的优点,为教学、科研及工业场景提供了标准化、智能化的参考解决方案。}

% 竖直间距
\vspace{0.5em}

% 中文关键词
\noindent\textbf{关键词:}\keywordscn{恒压过滤实验;图像识别,自动化数据采集与处理;数据可视化;Python软件开发}