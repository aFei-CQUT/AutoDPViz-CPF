%  ====================== 字体配置  ======================
%
%  配置文档使用的中英文字体及数学符号的显示效果
%
%  ====================== 说明栏目 ======================


% --- 中文排版字号命令集(原属于命令文件,但其逻辑和字体息息相关) ---

% 基于中文排版规范的字号定义(字号/行距)
\newcommand{\chuhao}{\fontsize{42pt}{50.4pt}\selectfont}      % 初号
\newcommand{\xiaochu}{\fontsize{36pt}{43.2pt}\selectfont}     % 小初
\newcommand{\yihao}{\fontsize{26pt}{31.2pt}\selectfont}       % 一号
\newcommand{\xiaoyi}{\fontsize{24pt}{28.8pt}\selectfont}      % 小一
\newcommand{\erhao}{\fontsize{22pt}{26.4pt}\selectfont}       % 二号
\newcommand{\xiaoer}{\fontsize{18pt}{21.6pt}\selectfont}      % 小二
\newcommand{\sanhao}{\fontsize{16pt}{19.2pt}\selectfont}      % 三号
\newcommand{\xiaosan}{\fontsize{15pt}{18pt}\selectfont}       % 小三
\newcommand{\sihao}{\fontsize{14pt}{16.8pt}\selectfont}       % 四号
\newcommand{\zhongsi}{\fontsize{13pt}{15.6pt}\selectfont}     % 中四
\newcommand{\xiaosi}{\fontsize{12pt}{14.4pt}\selectfont}      % 小四
\newcommand{\wuhao}{\fontsize{10.5pt}{15pt}\selectfont}       % 五号
\newcommand{\xiaowu}{\fontsize{9pt}{10.8pt}\selectfont}       % 小五
\newcommand{\liuhao}{\fontsize{7.5pt}{9pt}\selectfont}        % 六号
\newcommand{\xiaoliu}{\fontsize{6.5pt}{7.8pt}\selectfont}     % 小六


% ---------------------- 中文字体配置 ----------------------
% 主字体(宋体,对应\rmfamily命令)
\setCJKmainfont{SimSun}[
    ItalicFont = KaiTi,      % 斜体使用楷体(中文无真正斜体,需单独指定)
    BoldFont   = SimHei      % 粗体使用黑体(避免伪粗体失真)
]

% 无衬线字体(黑体,对应\sffamily命令)
\setCJKsansfont{SimHei}[
    AutoFakeBold = 3.0       % 伪粗体强度(0=禁用,1=轻微,3=最强)
]

% 等宽字体(仿宋,对应\ttfamily命令,用于代码/等宽文本)
\setCJKmonofont{FangSong}[
    AutoFakeSlant = 0.2      % 伪斜体倾斜度(0=直立,0.2=约10度倾斜)
]

% ---------------------- 英文字体配置 ----------------------
% 主字体(Times New Roman,学术论文标准字体)
\setmainfont{Times New Roman}[
    BoldFont = * Bold,       % 使用字体内置粗体(*表示继承主字体名)
    ItalicFont = * Italic    % 使用字体内置斜体
]

% 无衬线字体(Arial,用于标题/强调文本)
\setsansfont{Arial}[
    Scale = 0.95             % 轻微缩小以适应中文混排
]

% ---------------------- 数学字体配置 ----------------------
% XITS Math是Times风格的数学字体,与正文风格统一
\setmathfont{XITS Math}[
    Scale = MatchLowercase  % 大小与小写字母高度匹配
]

% 字号分级设置(确保数学符号与正文协调)
\DeclareMathSizes{10.5}{10.5}{7.875}{6.3}  % 正文/上标/下标字号比例

% ---------------------- 标题格式规范-本质上是字体配置 ----------------------


% 设置标题编号深度
\setcounter{secnumdepth}{5}

% 章节标题三级格式(字号/字体/间距)
\titleformat{\section}
{\xiaosi\rmfamily\bfseries \fontspec{Arial}\bfseries} % 中文四号宋体加粗,英文用Arial字号与中文一致
{\thesection}{1em}{}

\titleformat{\subsection}
{\wuhao\rmfamily\bfseries \fontspec{Arial}\bfseries}  % 中文五号宋体加粗,英文用Arial字号与中文一致
{\thesubsection}{1em}{}

\titleformat{\subsubsection}
{\wuhao\rmfamily\bfseries \fontspec{Arial}\bfseries}  % 中文五号宋体加粗,英文用Arial字号与中文一致
{\thesubsubsection}{1em}{}

% 四级标题
\titleclass{\subsubsubsection}{straight}[\subsubsection]
\newcounter{subsubsubsection}[subsubsection]
\renewcommand{\thesubsubsubsection}{\thesubsubsection.\arabic{subsubsubsection}}
\titleformat{\subsubsubsection}
  {\xiaowu\rmfamily\bfseries \fontspec{Arial}\bfseries}
  {\thesubsubsubsection}{1em}{}
\titlespacing*{\subsubsubsection}{0pt}{3.25ex plus 1ex minus .2ex}{1.5ex plus .2ex}
\makeatletter
\@addtoreset{subsubsubsection}{subsubsection}
\makeatother

% 五级标题
\titleclass{\subsubsubsubsection}{straight}[\subsubsubsection]
\newcounter{subsubsubsubsection}[subsubsubsection]
\renewcommand{\thesubsubsubsubsection}{\thesubsubsubsection.\arabic{subsubsubsubsection}}
\titleformat{\subsubsubsubsection}
  {\liuhao\rmfamily\bfseries \fontspec{Arial}\bfseries}
  {\thesubsubsubsubsection}{1em}{}
\titlespacing*{\subsubsubsubsection}{0pt}{3.25ex plus 1ex minus .2ex}{1.5ex plus .2ex}
\makeatletter
\@addtoreset{subsubsubsubsection}{subsubsubsection}
\makeatother

% ====================== 信息区块命令集 ======================

% -----------------------------------------------------------
% -本质上是字体配置
% ---------------------- 中文信息块命令 ----------------------
%
% -----------------------------------------------------------

% 中文标题(三号字,中文宋体加粗,英文Arial加粗)
\newcommand{\maketitlecn}[1]{\begin{center}\sanhao\rmfamily\bfseries \fontspec{Arial}\bfseries #1 \end{center}}

% 中文参赛选手
\newcommand{\authorscn}[1]{%
  \ifbool{HideForce}{}{%
    \ifbool{HideWeak}{%
      \coverinfo{参赛选手:}{#1}{\xiaosi\rmfamily}%
    }{%
      \begin{center}\xiaosi\rmfamily 参赛选手:#1\end{center}%
    }%
  }%
}

% 中文指导教师
\newcommand{\advisorcn}[1]{%
  \ifbool{HideForce}{}{%
    \ifbool{HideWeak}{%
      \coverinfo{指导教师:}{#1}{\xiaosi\rmfamily}%
    }{%
      \begin{center}\xiaosi\rmfamily 指导教师:#1\end{center}%
    }%
  }%
}

% 中文参赛学校与城市邮编
\newcommand{\schoolinfocn}[1]{%
  \ifbool{HideForce}{}{%
    \ifbool{HideWeak}{%
      \coverinfo[3.5cm]{城市邮编信息:}{#1}{\xiaowu\rmfamily}%
    }{%
      \begin{center}\xiaowu\rmfamily 城市邮编信息:#1\end{center}%
    }%
  }%
}

% 中文摘要小五号宋体,如果有英文小五号Arial
\newcommand{\abstractcn}[1]{\xiaowu\rmfamily \fontspec{Arial} #1}

% 中文关键词小五号宋体,如果有英文小五号Arial
\newcommand{\keywordscn}[1]{\xiaowu\rmfamily \fontspec{Arial} #1}

% -----------------------------------------------------------
% -本质上是字体配置
% ---------------------- 英文信息块命令 ----------------------
%
% -----------------------------------------------------------

% 英文标题 (Arial,四号加粗) - 实际上和中文的一致,只是没有中文字符罢了--\rmfamily\bfseries中文宋体加粗 -> 黑体
\newcommand{\maketitleen}[1]{\begin{center}\sanhao\rmfamily\bfseries \fontspec{Arial}\bfseries #1 \end{center}}

% 英文参赛选手
\newcommand{\authorsen}[1]{%
  \ifbool{HideForce}{}{%
    \ifbool{HideWeak}{%
      \coverinfo{\wuhao\fontspec{Arial} Participants:}{#1}{\wuhao\fontspec{Arial}}%
    }{%
      \begin{center}\wuhao\fontspec{Arial} Participants: #1\end{center}%
    }%
  }%
}

% 英文指导教师
\newcommand{\advisoren}[1]{%
  \ifbool{HideForce}{}{%
    \ifbool{HideWeak}{%
      \coverinfo{\wuhao\fontspec{Arial} Academic Advisor:}{#1}{\wuhao\fontspec{Arial}}%
    }{%
      \begin{center}\wuhao\fontspec{Arial} Academic Advisor: #1\end{center}%
    }%
  }%
}

% 英文单位信息
\newcommand{\schoolinfoen}[1]{%
  \ifbool{HideForce}{}{%
    \ifbool{HideWeak}{%
      \coverinfo[4.5cm]{\wuhao\fontspec{Arial} City and Postal Information:}{#1}{\xiaowu\fontspec{Arial}}%
    }{%
      \begin{center}\xiaowu\fontspec{Arial} City and Postal Information: #1\end{center}%
    }%
  }%
}

% 英文摘要小五号Arial
\newcommand{\abstracten}[1]{\xiaowu\fontspec{Arial} #1}

% 英文关键词小五号Arial
\newcommand{\keywordsen}[1]{\xiaowu\fontspec{Arial} #1}

% ---------------------- 图表系统规范-本质上是字体配置 ----------------------

% 图标题样式 + 表格标题样式(小五号加粗,居中)
\DeclareCaptionFont{tablefont}{\xiaowu\rmfamily\bfseries \fontspec{Times New Roman}}
\captionsetup{
  font      = {tablefont},
  labelfont = bf,
  labelsep  = quad,
  justification = centering
}

% 表格内部字体为6号字,中文宋体,英文Times New Roman
\newcommand{\settableinnerfont}{\liuhao\rmfamily \fontspec{Times New Roman}}