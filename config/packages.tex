% ====================== 加载库包  ======================

%  字体与编码系统
% \usepackage{ctex}             % 中文支持,提供中文排版功能,适用于中文文档,但不能与ctexart同时加载
% \usepackage{xeCJK}            % 也不需要加载xeCJK,ctex已包含中文字体支持

%  页面布局与样式
\usepackage{geometry}           % 页面尺寸与边距控制,便于设置页面大小和边距
\usepackage{setspace}           % 行距调整包,支持设置单倍行距、1.5倍行距等
\usepackage{titlesec}           % 章节标题格式定制,支持自定义章节标题样式
\usepackage[normalem]{ulem}     % 增强的下划线功能,提供如波浪线等多种下划线样式

%  表格相关库包
\usepackage{array}              % 扩展LaTeX表格列格式功能
\usepackage{booktabs}           % 专业的表格线规则,用于生成高质量的表格线,这里用来三线表
\usepackage{multirow}           % 多行表格定制
\usepackage{longtable}          % 跨页长表格支持,表格内容跨越多页时非常有用
\usepackage{tabularx}           % 自动调整列宽的表格,适合宽表格
\usepackage{threeparttable}     % 支持表格的标题、主体与脚注集成
\usepackage{makecell}           % 让表格单元格更灵活,支持多种样式配置
\usepackage{caption}            % 图表标题格式控制,可以修改标题字体、字号等
\usepackage{subcaption}        % 支持子图和子表格的标题格式控制

%  图片插入
\usepackage{graphicx}           % 图形插入支持,支持多种图片格式及尺寸调整

%  绘图库
\usepackage{tikz}
\usetikzlibrary{shapes.geometric, arrows.meta}
\usepackage{adjustbox}
\usepackage{xparse}

%  排版增强
\usepackage{ragged2e}           % 提升段落对齐功能,提供更灵活的对齐方式

%  数学符号字体
\usepackage{unicode-math}       % 用于设置数学字体这里我用了数学字体的格式所以要使用
\usepackage{fontspec}           % 用于设置字体,必须与 XeLaTeX 或 LuaLaTeX 配合使用

%  数学符号
\usepackage{amsmath}            % 数学公式增强,提供更多数学公式环境

%  其他学科支持
\usepackage{siunitx}            % 提供标准化的SI单位格式,支持自动格式化单位
\usepackage[version=4]{mhchem}  % 化学式排版,支持化学反应式、元素符号等排版
\usepackage{chemfig}            % chemfig 宏包,用于绘制化学结构式

%  参考文献与引用
\usepackage{cite}               % 引用优化,提供更灵活的参考文献引用格式

%  工具类与辅助宏包
\usepackage{enumitem}           % 列表格式定制,允许自定义项目符号、编号等格式
\usepackage{etoolbox}           % 提供更多LaTeX命令增强功能,简化自定义命令的创建
\usepackage{fancyhdr}           % 页眉页脚定制,允许用户自定义页眉和页脚的内容
\usepackage{xstring}

%  算法与程序代码
% \usepackage[linesnumbered, ruled]{algorithm2e} % 算法包与下面三个算法阵营有冲突

%  这三个组合加上浮动体支持跨页算法
\usepackage{algorithm}        % 算法浮动体环境
\usepackage{algpseudocode}    % 伪代码语法支持
\usepackage{float}            % 浮动体精确控制 (H 位置限定符)
\usepackage{lipsum}          % 占位文本生成 (调试用,正式文档可移除)